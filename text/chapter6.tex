

\chapter{More Lists}\label{CHAPTER6}

\Thischapter{145}{

\noindent
This chapter has two main goals:

\begin{enumerate}
\item{}To define ``append/3'', a predicate for concatenating two lists,
        and illustrate what can be done with it.

\item{}To discuss two ways of reversing a list: a naive method using
       ``append/3'', and a more efficient method using
        accumulators.

\end{enumerate}

}

\section{Append}\label{SEC.L6.APPEND}



We shall define an important predicate ``append/3" whose arguments are
all lists.  Viewed declaratively, ``append(L1,L2,L3)" will
\index{PROLOG append/3@\texttt{append/3}} hold when the list ``L3" is
the result of \LPNterm{concatena\-ting} the lists ``L1" and ``L2"
together (concatenating means joining the lists together, end to end).
For example, if we pose the query
\begin{LPNcodedisplay}
?- append([a,b,c],[1,2,3],[a,b,c,1,2,3]).
\end{LPNcodedisplay}
or the query
\begin{LPNcodedisplay}
?- append([a,[foo,gibble],c],[1,2,[[],b]],
          [a,[foo,gibble],c,1,2,[[],b]).
\end{LPNcodedisplay}
we will get the response ``yes''. On the other hand, if we pose
the query
\begin{LPNcodedisplay}
?- append([a,b,c],[1,2,3],[a,b,c,1,2]).
\end{LPNcodedisplay}
or the query
\begin{LPNcodedisplay}
?- append([a,b,c],[1,2,3],[1,2,3,a,b,c]).
\end{LPNcodedisplay}
we will get the response ``no''.


From a procedural perspective, the most obvious use of ``append/3"
is to concatenate two lists together. We can do this simply by using a
variable as the third argument: the query
\begin{LPNcodedisplay}
?- append([a,b,c],[1,2,3],L3).
\end{LPNcodedisplay}
yields the response
\begin{LPNcodedisplay}
L3 = [a,b,c,1,2,3]
yes
\end{LPNcodedisplay}

But (as we shall soon see) we can also use ``append/3" to split up a
list.  In fact, ``append/3" is a real workhorse.  There's lots we
can do with it, and studying it is a good way to gain a better
understanding of list processing in Prolog.

\subsection*{Defining append}\label{SUBSEC.L6.DEFINING.APPEND}



Here's how ``append/3" is defined:
\begin{LPNcodedisplay}
append([],L,L).
append([H|T],L2,[H|L3]) :- append(T,L2,L3).
\end{LPNcodedisplay}

This is a recursive definition. The base case simply says that
appending the empty list to any list whatsoever yields that same list,
which is obviously true.

But what about the recursive step?  This says that when we concatenate
a non-empty list ``[H|T]" with a list ``L2", we end up with
the list whose head is ``H" and whose tail is the result of
concatenating ``T" with ``L2". It may be useful to think about
this definition pictorially:

%\includegraphics{append.eps}

\begin{quote}
Input:  [ H $\mid$ \fbox{\ T \ } ] + \fbox{\ L2 \ } \\[5mm]
Result: [ H $\mid \underbrace{\fbox{\ L3 \ }}_{\fbox{T}+\fbox{L2}}$ ]
\end{quote}








But what is the procedural meaning of this definition?  What actually
goes on when we use ``append/3" to glue two lists together?
Let's take a detailed look at what happens when we pose the query
"?- append([a,b,c],[1,2,3],X)".

When we pose this query, Prolog will match it to the
head of the recursive rule, generating a new internal variable
(say ``\_G518") in the process. If we carried out a trace of what
happens next, we would get something like the following:

\begin{LPNcodedisplay}
append([a, b, c], [1, 2, 3], _G518)
append([b, c], [1, 2, 3], _G587)
append([c], [1, 2, 3], _G590)
append([], [1, 2, 3], _G593)
append([], [1, 2, 3], [1, 2, 3])
append([c], [1, 2, 3], [c, 1, 2, 3])
append([b, c], [1, 2, 3], [b, c, 1, 2, 3])
append([a, b, c], [1, 2, 3], [a, b, c, 1, 2, 3])

X = [a, b, c, 1, 2, 3]
yes
\end{LPNcodedisplay}

The basic pattern should be clear: in the first four lines we see that
Prolog recurses its way down the list in its first argument until it
can apply the base case of the recursive definition.  Then, as the
next four lines show, it then stepwise `fills in' the result.  How is
this `filling in' process carried out?  By successively instantiating
the variables ``\_G593", ``\_G590", ``\_G587", and
"\_G518". But while it's important to grasp this basic pattern, it
doesn't tell us all we need to know about the way ``append/3" works,
so let's dig deeper. Here is the search tree for the query
"append([a,b,c],[1,2,3],X)". We'll work carefully through
all the steps, making a careful note of what our goals are,
and what the variables are instantiated to.

\begin{center}
\includegraphics{chap6-pspic1.ps}
\end{center}


\begin{enumerate}

\item{}Goal 1: ``append([a,b,c],[1,2,3],\_G518)".  Prolog
matches this to the head of the recursive rule (that is,
"append([H|T],L2,[H|L3])").  Thus ``\_G518" is unified to
"[a|L3]", and Prolog has the new goal
"append([b,c],[1,2,3],L3)".
It generates a new variable ``\_G587" for
"L3", thus we have that ``\_G518 = [a|\_G587]".

\item{}Goal 2: ``append([b,c],[1,2,3],\_G587)".  Prolog matches
this to the head of the recursive rule, thus ``\_G587" is unified
to ``[b|L3]", and Prolog has the new goal
"append([c],[1,2,3],L3)".
It generates the internal variable ``\_G590" for
"L3", thus we have that ``\_G587 = [b|\_G590]".

\item{}Goal 3: ``append([c],[1,2,3],\_G590").  Prolog matches this
to the head of the recursive rule, thus ``\_G590" is unified  to
"[c|L3]", and Prolog has the new goal
"append([],[1,2,3],L3)".
It generates the internal variable ``\_G593" for ``L3",
thus we have that ``\_G590 = [c|\_G593]".

\item{}Goal 4: ``append([],[1,2,3],\_G593").  At last: Prolog
can use the base clause (that is, ``append([],L,L)").  And in the
four successive matching steps, Prolog will obtain answers to Goal 4,
Goal 3, Goal 2, and Goal 1. Here's how.

\item{}Answer to Goal 4: ``append([],[1,2,3],[1,2,3])". This
is because when we match Goal 4 (that is,
"append([],[1,2,3],\_G593)" to the base clause, ``\_G593" is
unified to ``[1,2,3]".

\item{}Answer to Goal 3: ``append([c],[1,2,3],[c,1,2,3])".  Why?
Because Goal 3 is ``append([c],[1,2,3],\_G590])", and ``\_G590" is the
list ``[c|\_G593]", and we have just unified ``\_G593" to ``[1,2,3]". So
"\_G590" is unified to ``[c,1,2,3]".

\item{}Answer to Goal 2: ``append([b,c],[1,2,3],[b,c,1,2,3])".
Why? Because Goal 2 is ``append([b,c],[1,2,3],\_G587])", and
"\_G587" is the list  ``[b|\_G590]", and we have just unified ``\_G590" to
"[c,1,2,3]". So ``\_G587" is unified to ``[b,c,1,2,3]".

\item{}Answer to Goal 1: ``append([a,b,c],[1,2,3],[a,b,c,1,2,3])".
Why? Because Goal 2 is ``append([a,b,c],[1,2,3],\_G518])",  and
"\_G518" is the list ``[a|\_G587]", and we have just unified ``\_G587" to
"[b,c,1,2,3]". So ``\_G518" is unified to ``[a,b,c,1,2,3]".

\item{}Thus Prolog now knows how to instantiate ``X", the original
query variable.  It tells us that ``X = [a,b,c,1,2,3]", which
is what we want.
\end{enumerate}


Work through this example carefully, and make sure you fully
understand the pattern of variable instantiations, namely:
\begin{LPNcodedisplay}
_G518 = [a|_G587]
      = [a|[b|_G590]]
      = [a|[b|[c|_G593]]]
\end{LPNcodedisplay}
This type of pattern lies at the heart of the way ``append/3" works.
Moreover, it illustrates a more general theme: the use of unification to
build structure. In a nutshell, the recursive calls to ``append/3" build up
this nested pattern of variables which code up the required
answer. When Prolog finally instantiates the innermost variable
"\_G593" to ``[1, 2, 3]", the answer crystallises out, like a snowflake
forming around a grain of dust. But it is unification, not magic, that
produces the result.



\subsection*{Using append}\label{SUBSEC.L6.USING.APPEND}



Now that we understand how ``append/3" works,
let's see how we can put it to work.

One important use of ``append/3" is to split up a list into two
consecutive lists.  For example:
\begin{LPNcodedisplay}
?- append(X,Y,[a,b,c,d]).

X = []
Y = [a,b,c,d] ;

X = [a]
Y = [b,c,d] ;

X = [a,b]
Y = [c,d] ;

X = [a,b,c]
Y = [d] ;

X = [a,b,c,d]
Y = [] ;

no
\end{LPNcodedisplay}

That is, we give the list we want to split up (here"[a,b,c,d]")
to ``append/3" as the third argument, and we use variables for the
first two arguments.  Prolog then searches for ways of instantiating
the variables to two lists that concatenate to give the third
argument, thus splitting up the list in two. Moreover, as this example
shows, by backtracking, Prolog can find all possible ways of splitting
up a list into two consecutive lists.

This ability means it is easy to define some useful predicates with
"append/3".  Let's consider some examples.  First, we can define a
program which finds \LPNterm{prefixes} of lists.  For example, the
prefixes of ``[a,b,c,d]" are ``[]", ``[a]", ``[a,b]",
"[a,b,c]", and ``[a,b,c,d]".  With the help of ``append/3"
it is straightforward to define a program ``prefix/2", whose
arguments are both lists, such that ``prefix(P,L)" will hold when
"P" is a prefix of ``L".  Here's how:
\begin{LPNcodedisplay}
prefix(P,L):- append(P,_,L).
\end{LPNcodedisplay}

This says that list ``P" is a prefix of list ``L" when there
is some list such that ``L" is the result of concatenating
"P" with that list.  (We use the anonymous variable since we
don't care what that other list is: we only care that there is some such
list or other.) This predicate successfully finds prefixes of lists,
and moreover, via backtracking, finds them all:
\begin{LPNcodedisplay}
?- prefix(X,[a,b,c,d]).

X = [] ;

X = [a] ;

X = [a,b] ;

X = [a,b,c] ;

X = [a,b,c,d] ;

no
\end{LPNcodedisplay}


In a similar fashion, we can define a program which finds
\LPNterm{suffixes} of lists.  For example, the suffixes of
"[a,b,c,d]" are ``[]", ``[d]", ``[c,d]",
"[b,c,d]", and ``[a,b,c,d]".  Again, using ``append/3" it
is easy to define ``suffix/2", a predicate whose arguments
are both lists, such that ``suffix(S,L)" will hold when ``S"
is a suffix of ``L":
\begin{LPNcodedisplay}
suffix(S,L):- append(_,S,L).
\end{LPNcodedisplay}
That is, list ``S" is a suffix of list ``L" if there is some
list such that ``L" is the result of concatenating that list with
"S".  This predicate successfully finds suffixes of lists, and
moreover, via backtracking, finds them all:
\begin{LPNcodedisplay}
?- suffix(X,[a,b,c,d]).

X = [a,b,c,d] ;

X = [b,c,d] ;

X = [c,d] ;

X = [d] ;

X = [] ;

no
\end{LPNcodedisplay}

Make sure you understand why the results come out in this order.

\bigskip

And now it's very easy to define a program that finds
\LPNterm{sublists} of lists.  The sublists of ``[a,b,c,d]" are ``[]",
"[a]", ``[b]", ``[c]", ``[d]", ``[a,b]",
"[b,c]", ``[c,d]", ``[a,b,c]",
"[b,c,d]", and ``[a,b,c,d]".  A little thought reveals
that the sublists of a list L are simply the \LPNemph{prefixes of suffixes
of} L. Think about it pictorially:


\begin{quote}
Take suffix: \fbox{$a,b,c,d,e,f,g,\underbrace{h,i,j,k,l,m,n,o,p}$}\\[3mm]
Take prefix: \hspace*{23mm}\fbox{$\underbrace{h,i,j,k,l},m,n,o,p$}\\[3mm]
Result:     \hspace*{31mm}\fbox{$h,i,j,k,l$}
\end{quote}


As we already have defined the predicates for producing suffixes and prefixes
of lists, we simply define a sublist as:
\begin{LPNcodedisplay}
sublist(SubL,L):- suffix(S,L), prefix(SubL,S).
\end{LPNcodedisplay}
That is, ``SubL" is a sublist of ``L" if there is some suffix ``S" of ``L"
of which ``SubL" is a prefix.  This program doesn't
\LPNemph{explicitly} use ``append/3", but of course, under the surface,
that's what's doing the work for us, as both ``prefix/2" and ``suffix/2" are
defined using ``append/3".




\section{Reversing a List}\label{SEC.L6.REVERSE}

The ``append/3" predicate is useful, and it is important to know how to
put it to work. But it is just as important to know that it can be a
source of inefficiency, and that you probably don't want to use it all
the time.

Why is ``append/3" a source of inefficiency?  If you think about
the way it works, you'll notice a weakness: ``append/3" doesn't join
two lists in one simple action. Rather, it needs to work its way down
its first argument until it finds the end of the list, and only then
can it carry out the concatenation.

Now, often this causes no problems. For example, if we have two lists
and we just want to concatenate them, it's probably not too bad. Sure,
Prolog will need to work down the length of the first list, but if the
list is not too long, that's probably not too high a price
to pay for the ease of working with ``append/3".

But matters may be very different if the first two arguments are given
as variables. As we've just seen, it can be very useful to give
"append/3" variables in its first two arguments, for this lets
Prolog search for ways of splitting up the lists. But there is a price
to pay: a lot of searching is going on, and this can lead to very
inefficient programs.

To illustrate this, we shall examine the problem of reversing a list.
That is, we will examine the problem of defining a predicate which
takes a list (say ``[a,b,c,d]") as input and returns a list
containing the same elements in the reverse order (here
``[d,c,b,a]").

Now, a reverse predicate is a useful predicate to have around. As you
will have realised by now, lists in Prolog are far easier to access
from the front than from the back. For example, to pull out the head
of a list ``L", all we have to do is perform the unification
``[H|\_] = L"; this results in ``H" being instantiated to the head of ``L". But
pulling out the last element of an arbitrary list is harder: we can't
do it simply using unification.  On the other hand, if we had a
predicate which reversed lists, we could first reverse the input list,
and then pull out the head of the reversed list, as this would give us
the last element of the original list.  So a reverse predicate could
be a useful tool.  However, as we may have to reverse large lists, we
would like this tool to be efficient.  So we need to think about the
problem carefully.

And that's what we're going to do now.  We will define two reverse
predicates: a naive one, defined with the help of ``append/3", and a
more efficient (and indeed, more natural) one defined using
accumulators.

\subsection*{Naive reverse using append}\label{SUBSEC.L6.REVERSE.NAIVE}



Here's a recursive definition of what is involved in reversing a list:

\begin{enumerate}
\item{}If we reverse the empty list, we obtain the
        empty list.
\item{}If we reverse the list ``[H|T]", we end up with the list
        obtained by reversing ``T" and concatenating with ``[H]".
\end{enumerate}
To see that the recursive clause is correct, consider the list
``[a,b,c,d]".  If we reverse the tail of this list we obtain
``[d,c,b]".  Concatenating this with ``[a]" yields
``[d,c,b,a]", which is the reverse of ``[a,b,c,d]".

With the help of ``append/3" it is easy to turn this recursive
definition into Prolog:

\begin{LPNcodedisplay}
naiverev([],[]).
naiverev([H|T],R):- naiverev(T,RevT), append(RevT,[H],R).
\end{LPNcodedisplay}

Now, this definition is correct, but it does an awful lot of work.
It is \LPNemph{very} instructive to look at a trace of this program.  This
shows that the program is spending a lot of time carrying out appends.
This shouldn't be too surprising: after, all, we are calling
``append/3" recursively.  The result is very
inefficient (if you run a trace, you will find that it takes about 90
steps to reverse an eight element list) and hard to understand (the
predicate spends most of its time in the recursive calls to
``append/3", making it very hard to see what is going on).

Not nice. But as we shall now see, there \LPNemph{is} a better way.



\subsection*{Reverse using an accumulator}\label{SUBSEC.L6.REVERSE.ACC}

The better way is to use an accumulator.  The underlying idea is
simple and natural.  Our accumulator will be a list, and when we start
it will be empty.  Suppose we want to reverse ``[a,b,c,d]".  At
the start, our accumulator will be ``[]".  So we simply take the
head of the list we are trying to reverse and add it as the head of
the accumulator.  We then carry on processing the tail, thus we are
faced with the task of reversing ``[b,c,d]", and our accumulator
is ``[a]".  Again we take the head of the list we are trying to
reverse and add it as the head of the accumulator (thus our new
accumulator is ``[b,a]") and carry on trying to reverse ``[c,d]".
Again we use the same idea, so we get a new accumulator
``[c,b,a]", and try to reverse ``[d]".  Needless to say, the
next step yields an accumulator ``[d,c,b,a]" and the new goal of
trying to reverse ``[]".  This is where the process stops: \LPNemph{and
our accumulator contains the reversed list we want}.  To summarise:
the idea is simply to work our way through the list we want to
reverse, and push each element in turn onto the head of the
accumulator, like this:
\begin{LPNcodedisplay}
List: [a,b,c,d]  Accumulator: []
List: [b,c,d]    Accumulator: [a]
List: [c,d]      Accumulator: [b,a]
List: [d]        Accumulator: [c,b,a]
List: []         Accumulator: [d,c,b,a]
\end{LPNcodedisplay}
This will be efficient because we simply blast our way through the list once:
we don't have to waste time carrying out concatenation or other
irrelevant work.

It's also easy to put this idea in Prolog.  Here's the accumulator
code:
\begin{LPNcodedisplay}
accRev([H|T],A,R):- accRev(T,[H|A],R).
accRev([],A,A).
\end{LPNcodedisplay}
This is classic accumulator code: it follows the same pattern
as the arithmetic examples we examined in the previous chapter.  The
recursive clause is responsible for chopping off the head of the input
list, and pushing it onto the accumulator.  The base case halts the
program, and copies the accumulator to the final argument.

As is usual with accumulator code, it's a good idea to write a
predicate which carries out the required initialisation of the
accumulator for us:
\begin{LPNcodedisplay}
rev(L,R):- accRev(L,[],R).
\end{LPNcodedisplay}

Again, it is instructive to run some traces on this program and
compare it with ``naiverev/2".  The accumulator based version is
\LPNemph{clearly} better. For example, it takes about 20 steps to reverse
an eight element list, as opposed to 90 for the naive version.
Moreover, the trace is far easier to follow. The idea underlying the
accumulator based version is simpler and more natural than the
recursive calls to ``append/3".


Summing up, ``append/3" is a useful program, and you certainly should not
be scared of using it. However, you also need to be aware that it is a
source of inefficiency, so when you use it, ask yourself whether there
is a better way.  And often there is. The use of accumulators is often
better, and (as the ``rev/2" example show) accumulators can be a
natural way of handling list processing tasks.

\section{Exercises}\label{SEC.L6.EXERCISES}


\begin{LPNexercise}{L6.EX1}Let's call a list \LPNemph{doubled}
if it is made of two consecutive blocks of elements that are
exactly the same.
For example, ``[a,b,c,a,b,c]" is doubled (it's
made up of ``[a,b,c]" followed by ``[a,b,c]")
and so is ``[foo,gubble,foo,gubble]". On the other hand,
``[foo,gubble,foo]" is not doubled. Write a predicate
``doubled(List)" which succeeds when ``List"
is a doubled list.\end{LPNexercise}
\begin{LPNexercise}{L6.EX2}A palindrome is a word or phrase that spells the same forwards and
backwards.  For example, `rotator', `eve', and `nurses run' are all
palindromes.  Write a predicate ``palindrome(List)", which checks
whether ``List" is a palindrome. For example, to the queries
\begin{LPNcodedisplay}
?- palindrome([r,o,t,a,t,o,r]).
\end{LPNcodedisplay}
and
\begin{LPNcodedisplay}
?- palindrome([n,u,r,s,e,s,r,u,n]).
\end{LPNcodedisplay}
Prolog should respond ``yes'', but to the query
\begin{LPNcodedisplay}
?- palindrome([n,o,t,h,i,s]).
\end{LPNcodedisplay}
it should respond ``no''.\end{LPNexercise}



\begin{LPNexercise}{L6.EX3c}
Write a predicate ``toptail(InList,OutList)" which says
no if ``InList" is a list containing fewer than 2 elements, and
which deletes the first and the last elements of ``InList" and
returns the result as ``OutList", when ``InList" is a
list containing at least 2 elements. For example:
\begin{LPNcodedisplay}
    toptail([a],T).
    no

    toptail([a,b],T).
    T=[]

    toptail([a,b,c],T).
    T=[b]
\end{LPNcodedisplay}

(Hint: here's where ``append/3" comes in useful.)
\end{LPNexercise}

\begin{LPNexercise}{L6.EX3d}
Write a predicate ``last(List,X)" which is true only when ``List" is a
list that contains at least one element and ``X" is the last element of
that list. Do this in two different ways:
\begin{enumerate}
\item Define ``last/2" using the predicate ``rev/2" discussed in the text.
\item Define ``last/2" using recursion.
\end{enumerate}
\end{LPNexercise}

\begin{LPNexercise}{L6.EX3e}
Write a predicate ``swapfl(List1,List2)" which checks whether ``List1"
is identical to ``List2", except that the first and last elements are
exchanged.  Here's where ``append/3" could come in useful again, but it
is also possible to write a recursive definition without appealing to
"append/3" (or any other) predicates.
\end{LPNexercise}





\begin{LPNexercise}{L6.EX4}
Here is an exercise for those of you who like logic
puzzles.

There is a street with three neighbouring houses that all have a
different colour, namely red, blue, and green. People of different
nationalities live in the different houses and they all have a
different pet. Here are some more facts about them:
\begin{itemize}
\item{}The Englishman lives in the red house.
\item{}The jaguar is the pet of the Spanish family.
\item{}The Japanese lives to the right of the snail keeper.
\item{}The snail keeper lives to the left of the blue house.
\end{itemize}
Who keeps the zebra? Don't work it out for yourself:
define a predicate ``zebra/1" that tells you the nationality
of the owner of the zebra!

(Hint: Think of a representation for the houses and the street. Code the
four constraints in Prolog. You may find ``member/2" and ``sublist/2"
useful.)
\end{LPNexercise}


\section{Practical Session}\label{SEC.L6.PRAXIS}



The purpose of Practical Session 6 is to help you get more experience
with list manipulation.  We first suggest some traces for you to carry
out, and then some programming exercises.

The following traces will help you get to grips with the
predicates  discussed in the text:

1. Carry out traces of ``append/3" with the
first two arguments instantiated, and the third argument
uninstantiated. For example, ``append([a,b,c],[[],[2,3],b],X)"
Make sure the basic pattern is clear.

2. Next, carry out traces on ``append/3" as used to split up a
list, that is, with the first two arguments given as variables, and the
last argument instantiated. For example,
``append(L,R,[foo,wee,blup])."

3. Carry out some traces on ``prefix/2" and ``suffix/2". Why
does ``prefix/2" find shorter lists first, and
``suffix/2" longer lists first?

4. Carry out some traces on ``sublist/2". As we said in the text,
via backtracking this predicate generates all possible sublists, but
as you'll see, it generates several sublists more than once. Do you
understand why?

5. Carry out traces on both ``naiverev/2" and ``rev/2", and
compare their behaviour.


Now for some programming work:


1. It is possible to write a one line definition of the
``member" predicate by making use of ``append/3". Do so. How
does this new version of ``member" compare in efficiency with the
standard one?
2. Write a predicate ``set(InList,OutList)" which takes as
input an arbitrary list, and returns a list in which each element of
the input list appears only once. For example, the query
\begin{LPNcodedisplay}
?- set([2,2,foo,1,foo, [],[]],X).
\end{LPNcodedisplay}

should yield the result
\begin{LPNcodedisplay}
X = [2,foo,1,[]].
\end{LPNcodedisplay}

(Hint: use  the ``member" predicate to test for repetitions
of items you have already found.)

3. We `flatten' a list by removing all the square brackets around
any lists it contains as elements, and around any lists that its
elements contain as elements, and so on, for all nested lists. For
example, when we flatten the list
\begin{LPNcodedisplay}
[a,b,[c,d],[[1,2]],foo]
\end{LPNcodedisplay}

we get the list
\begin{LPNcodedisplay}
[a,b,c,d,1,2,foo]
\end{LPNcodedisplay}

and when we flatten the list
\begin{LPNcodedisplay}
[a,b,[[[[[[[c,d]]]]]]],[[1,2]],foo,[]]
\end{LPNcodedisplay}
we also get
\begin{LPNcodedisplay}
[a,b,c,d,1,2,foo].
\end{LPNcodedisplay}
Write a predicate ``flatten(List,Flat)" that holds when the first
argument ``List" flattens to the second argument ``Flat".
This should be done without making use of ``append/3".


\medskip

Ok, we're now halfway through the book. And flattening a list is the
Pons Asinorum of Prolog programming. Did you cross it OK?  If so,
great. Time to move on.
