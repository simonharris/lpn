
\chapter{Facts, Rules, and Queries}\label{CHAPTER1}


\Thischapter{200}{

\noindent This chapter has two main goals:

\begin{enumerate}

\item{}To give some simple examples of Prolog programs. This will
  introduce us to the three basic constructs in Prolog:
  facts, rules, and queries. It will also
  introduce us to a number of other themes, like the role of
  logic in Prolog, and the idea of performing
  unification with the aid of variables.

\item{}To begin the systematic study of Prolog by defining
  terms, atoms, variables and other syntactic concepts.

\end{enumerate}

}


\section{Some Simple Examples}\label{SEC.L1.SIMPLE.EX}



There are only three basic constructs in Prolog: facts, rules, and
queries.  A collection of facts and rules is called a
\LPNterm{knowledge base} (or a \LPNterm{database}) and Prolog
programming is all about writing knowledge bases. That is, Prolog
programs simply \LPNemph{are} knowledge bases, collections of facts
and rules which describe some collection of relationships that we find
interesting.

So how do we \LPNemph{use} a Prolog program? By posing queries. That
is, by asking questions about the information stored in the knowledge
base.

Now this probably sounds rather strange. It's certainly not obvious
that it has much to do with programming at all. After all, isn't
programming all about telling a computer what to do?  But as we shall
see, the Prolog way of programming makes a lot of sense, at least for
certain tasks; for example, it is useful in computational linguistics
and Artificial Intelligence (AI).  But instead of saying more about
Prolog in general terms, let's jump right in and start writing some
simple knowledge bases; this is not just the best way of learning
Prolog, it's the only way.



\subsection*{Knowledge Base 1}\label{SUBSEC.L1.KB1}


Knowledge Base 1 (KB1) is simply a collection of \LPNterm{facts}.
Facts are used to state things that are \LPNemph{unconditionally} true
of some situation of interest.  For example, we can state that Mia,
Jody, and Yolanda are women, that Jody plays air guitar, and that a
party is taking place, using the following five facts:

\begin{LPNcodedisplay}
woman(mia).
woman(jody).
woman(yolanda).
playsAirGuitar(jody).
party.
\end{LPNcodedisplay}


This collection of facts is KB1. It is our first example of a Prolog program.
Note that the names "mia", "jody", and "yolanda", the properties "woman" and
"playsAirGuitar", and the proposition "party" have been written so that the
first letter is in lower-case.  This is important; we will see why a little
later on.

How can we use KB1?  By posing \LPNterm{queries}. That is, by asking questions
about the information KB1 contains.  Here are some examples.  We can ask Prolog
whether Mia is a woman by posing the query:
\begin{LPNcodedisplay}
?- woman(mia).
\end{LPNcodedisplay}
Prolog will answer
\begin{LPNcodedisplay}
yes
\end{LPNcodedisplay}
for the obvious reason that this is one of the facts explicitly
recorded in KB1.  Incidentally, \LPNemph{we} don't type in the "?-".
This symbol (or something like it, depending on the implementation of
Prolog you are using) is the prompt symbol that the Prolog interpreter
displays when it is waiting to evaluate a query.  We just type in the
actual query (for example "woman(mia)") followed by "." (a full stop).
The full stop is important. If you don't type it, Prolog won't start
working on the query.

Similarly, we can ask whether Jody plays air guitar by posing the
following query:

\begin{LPNcodedisplay}
?- playsAirGuitar(jody).
\end{LPNcodedisplay}
Prolog will again answer yes, because this is one of the facts in
KB1.  However, suppose we ask whether Mia plays air guitar:
\begin{LPNcodedisplay}
?- playsAirGuitar(mia).
\end{LPNcodedisplay}
We will get the answer
\begin{LPNcodedisplay}
no
\end{LPNcodedisplay}
Why? Well, first of all, this is not a fact in KB1. Moreover, KB1 is
extremely simple, and contains no other information (such as the
\LPNemph{rules} we will learn about shortly) which might help Prolog
try to \LPNterm{infer} (that is, \LPNterm{deduce}) whether Mia plays air
guitar. So Prolog correctly concludes that "playsAirGuitar(mia)" does
\LPNemph{not} follow from KB1.

Here are two important examples. First, suppose we pose the query:
\begin{LPNcodedisplay}
?- playsAirGuitar(vincent).
\end{LPNcodedisplay}
Again Prolog answers no. Why? Well, this query is about a person (Vincent)
that it has no information about, so it (correctly) concludes that
"playsAirGuitar(vincent)" cannot be deduced from the information in KB1.

Similarly, suppose we pose the query:
\begin{LPNcodedisplay}
?- tatooed(jody).
\end{LPNcodedisplay}
Again Prolog will answer no. Why? Well, this query is about a property
(being tatooed) that it has no information about, so once again it
(correctly) concludes that the query cannot be deduced from the
information in KB1.  (Actually, some Prolog implementations will
respond to this query with an error message, telling you that the
predicate or procedure "tatooed" is not defined; we will soon
introduce the notion of predicates.)

Needless to say, we can also make queries concerning propositions. For example,
if we pose the query
\begin{LPNcodedisplay}
?- party.
\end{LPNcodedisplay}
then Prolog will respond
\begin{LPNcodedisplay}
yes
\end{LPNcodedisplay}
and if we pose the query
\begin{LPNcodedisplay}
?- rockConcert.
\end{LPNcodedisplay}
then Prolog will respond
\begin{LPNcodedisplay}
no
\end{LPNcodedisplay}
exactly as we would expect.

\subsection*{Knowledge Base 2}\label{SUBSEC.L1.KB2}

Here is KB2, our second knowledge base:

\begin{LPNcodedisplay}
happy(yolanda).
listens2Music(mia).
listens2Music(yolanda):- happy(yolanda).
playsAirGuitar(mia):- listens2Music(mia).
playsAirGuitar(yolanda):- listens2Music(yolanda).
\end{LPNcodedisplay}
There are two facts in KB2, "listens2Music(mia)" and "happy(yolanda)".
The last three items it contains are \LPNterm{rules}.

Rules state information that is \LPNemph{conditionally} true of the
situation of interest.  For example, the first rule says that Yolanda
listens to music \LPNemph{if} she is happy, and the last rule says
that Yolanda plays air guitar \LPNemph{if} she listens to music. More
generally, the ":-" should be read as ``if'', or ``is implied by''.
The part on the left hand side of the ":-" is called the \LPNterm{head}
of the rule, the part on the right hand side is called the
\LPNterm{body}. So in general rules say: \LPNemph{if} the body of the
rule is true, \LPNemph{then} the head of the rule is true too.  And
now for the key point:
\begin{quote}
\LPNemph{If a knowledge base contains a rule}
"head :- body," \LPNemph{and Prolog knows that} "body"
\LPNemph{follows from the information in the knowledge base, then
  Prolog can infer} "head."
\end{quote}
This fundamental deduction step is
called \LPNterm{modus ponens}.

Let's consider an example. Suppose we ask whether Mia plays air
guitar:
\begin{LPNcodedisplay}
?- playsAirGuitar(mia).
\end{LPNcodedisplay}
Prolog will respond yes. Why?  Well, although it can't find
"playsAirGuitar(mia)" as a fact explicitly recorded in KB2, it can find the
rule
\begin{LPNcodedisplay}
playsAirGuitar(mia):- listens2Music(mia).
\end{LPNcodedisplay}
Moreover, KB2 also contains the fact "listens2Music(mia)".  Hence
Prolog can use the rule of modus ponens to deduce that
"playsAirGuitar(mia)".

Our next example shows that Prolog can chain together uses of modus
ponens. Suppose we ask:
\begin{LPNcodedisplay}
?- playsAirGuitar(yolanda).
\end{LPNcodedisplay}
Prolog would respond yes.  Why? Well, first of all, by using the fact
"happy(yolanda)" and the rule
\begin{LPNcodedisplay}
listens2Music(yolanda):- happy(yolanda).
\end{LPNcodedisplay}
Prolog can deduce the new fact "listens2Music(yolanda)".  This new
fact is not explicitly recorded in the knowledge base --- it is only
\LPNemph{implicitly} present (it is \LPNemph{inferred} knowledge).
Nonetheless, Prolog can then use it just like an explicitly recorded
fact. In particular, from this inferred fact and the rule
\begin{LPNcodedisplay}
playsAirGuitar(yolanda):- listens2Music(yolanda).
\end{LPNcodedisplay}
it can deduce "playsAirGuitar(yolanda)", which is what we asked it.
Summing up: any fact produced by an application of modus ponens can be used as
input to further rules. By chaining together applications of modus ponens in
this way, Prolog is able to retrieve information that logically follows from
the rules and facts recorded in the knowledge base.

The facts and rules contained in a knowledge base are called
\LPNterm{clauses}.  Thus KB2 contains five clauses, namely three rules
and two facts. Another way of looking at KB2 is to say that it
consists of three \LPNterm{predicates} (or \LPNterm{procedures}). The
three predicates are:
\begin{LPNcodedisplay}
listens2Music
happy
playsAirGuitar
\end{LPNcodedisplay}
The "happy" predicate is defined using a single clause (a fact). The
"listens2Music" and "playsAirGuitar" predicates are each defined using
two clauses (in one case, two rules, and in the other case, one rule
and one fact).  It is a good idea to think about Prolog programs in
terms of the predicates they contain.  In essence, the predicates are
the concepts we find important, and the various clauses we write down
concerning them are our attempts to pin down what they mean and how
they are inter-related.

One final remark. We can view a fact as a rule with an empty body.
That is, we can think of facts as conditionals that do not have any
antecedent conditions, or degenerate rules.



\subsection*{Knowledge Base 3}\label{SUBSEC.L1.KB3}



KB3, our third knowledge base, consists of five clauses:

\begin{LPNcodedisplay}
happy(vincent).
listens2Music(butch).
playsAirGuitar(vincent):-
   listens2Music(vincent),
   happy(vincent).
playsAirGuitar(butch):-
   happy(butch).
playsAirGuitar(butch):-
   listens2Music(butch).
\end{LPNcodedisplay}
There are two facts,  "happy(vincent)" and
"listens2Music(butch)", and three rules.


KB3 defines the same three predicates as KB2 (namely "happy",
"listens2Music", and "playsAirGuitar") but it defines them
differently.  In particular, the three rules that define the
"playsAirGuitar" predicate introduce some new ideas. First, note that
the rule\index{PROLOG ,/2@\texttt{,/2}}
\begin{LPNcodedisplay}
playsAirGuitar(vincent):-
   listens2Music(vincent),
   happy(vincent).
\end{LPNcodedisplay}
has \LPNemph{two} items in its body, or (to use the standard
terminology) two \LPNterm{goals}.  So, what exactly does this rule
mean?  The most important thing to note is the comma "," that
separates the goal "listens2Music(vincent)" and the goal
"happy(vincent)" in the rule's body.  This is the way logical
\LPNterm{conjunction} is expressed in Prolog (that is, the comma means
\LPNemph{and\/}).  So this rule says: ``Vincent plays air guitar if he
listens to music \LPNemph{and} he is happy''.

Thus, if we posed the query
\begin{LPNcodedisplay}
?- playsAirGuitar(vincent).
\end{LPNcodedisplay}
Prolog would answer no.  This is because while KB3 contains
"happy(vincent)", it does \LPNemph{not} explicitly contain the
information "listens2Music(vincent)", and this fact cannot be deduced
either.  So KB3 only fulfils one of the two preconditions needed to
establish "playsAirGuitar(vincent)", and our query fails.

Incidentally, the spacing used in this rule is irrelevant. For
example, we could have written it as
\begin{LPNcodedisplay}
playsAirGuitar(vincent):- listens2Music(vincent),
			   happy(vincent).
\end{LPNcodedisplay}
and it would have meant exactly the same thing. Prolog offers us a lot
of freedom in the way we set out knowledge bases, and we can take
advantage of this to keep our code readable.

Next, note that KB3 contains two rules with \LPNemph{exactly} the same
head, namely:
\begin{LPNcodedisplay}
playsAirGuitar(butch):-
   happy(butch).
playsAirGuitar(butch):-
   listens2Music(butch).
\end{LPNcodedisplay}
This is a way of stating that Butch plays air guitar
\LPNemph{either} if he listens to music, \LPNemph{or} if he is happy.
That is, listing multiple rules with the same head is a way of
expressing logical \LPNterm{disjunction} (that is, it is a way of
saying \LPNemph{or}).  So if we posed the query
\begin{LPNcodedisplay}
?- playsAirGuitar(butch).
\end{LPNcodedisplay}
Prolog would answer yes. For although the first of these rules
will not help (KB3 does not allow Prolog to conclude that
"happy(butch)"), KB3 \LPNemph{does} contain "listens2Music(butch)"
and this means Prolog can apply modus ponens using the rule
\begin{LPNcodedisplay}
playsAirGuitar(butch):-
   listens2Music(butch).
\end{LPNcodedisplay}
to conclude that "playsAirGuitar(butch)".

There is another way of expressing disjunction in Prolog. We could
replace the pair of rules given above by the single rule\index{PROLOG
;/2@\texttt{;/2}}
\begin{LPNcodedisplay}
playsAirGuitar(butch):-
   happy(butch);
   listens2Music(butch).
\end{LPNcodedisplay}
That is, the semicolon ";" is the Prolog symbol for \LPNemph{or}, so
this single rule means exactly the same thing as the previous pair of
rules. Is it better to use multiple rules or the semicolon?  That
depends. On the one hand, extensive use of semicolon can make Prolog
code hard to read. On the other hand, the semicolon is more efficient
as Prolog only has to deal with one rule.

It should now be clear that Prolog has something to do with
\LPNterm{logic}: after all, the ":-" means implication, the "," means
conjunction, and the ";" means disjunction.  (What about negation?
That is a whole other story. We'll be discussing it in
Chapter~\ref{CHAPTER10}.)  Moreover, we have seen that a standard
logical proof rule (modus ponens) plays an important role in Prolog
programming.  So we are already beginning to understand why ``Prolog''
is short for ``Programming with logic''.



\subsection*{Knowledge Base 4}\label{SUBSEC.L1.KB4}

Here is KB4, our fourth knowledge base:

\begin{LPNcodedisplay}
woman(mia).
woman(jody).
woman(yolanda).

loves(vincent,mia).
loves(marsellus,mia).
loves(pumpkin,honey_bunny).
loves(honey_bunny,pumpkin).
\end{LPNcodedisplay}


Now, this is a pretty boring knowledge base. There are no rules, only
a collection of facts. Ok, we are seeing a relation that has two names
as arguments for the first time (namely the "loves" relation), but,
let's face it, that's a rather predictable idea.

No, the novelty this time lies not in the knowledge base, it lies in
the queries we are going to pose. In particular, \LPNemph{for the
first time we're going to make use of \LPNterm{variables}}.  Here's an
example:
\begin{LPNcodedisplay}
?- woman(X).
\end{LPNcodedisplay}

The "X" is a variable (in fact, any word beginning with an upper-case
letter is a Prolog variable, which is why we had to be careful to use
lower-case initial letters in our earlier examples).  Now a variable
isn't a name, rather it's a \LPNemph{placeholder} for
information. That is, this query asks Prolog: tell me which of the
individuals you know about is a woman.

\pagebreak   %%% NEEDED TO GET MARGINPAR IN RIGHT MARGIN (LATEX BUG)
Prolog answers this query by working its way through KB4, from top to
bottom, trying to \LPNterm{unify} (or \LPNterm{match}) the expression
"woman(X)" with the information KB4 contains. Now the first item in
the knowledge base is "woman(mia)". So, Prolog unifies "X" with "mia",
thus making the query agree perfectly with this first item.
(Incidentally, there's a lot of different terminology for this
process: we can also say that Prolog \LPNterm{instantiates} "X" to
"mia", or that it \LPNterm{binds} "X" to "mia".)  Prolog then reports
back to us as follows:
\begin{LPNcodedisplay}
X = mia
\end{LPNcodedisplay}
That is, it not only says that there is information about at least one
woman in KB4, it actually tells us who she is. It didn't just say yes,
it actually gave us the \LPNterm{variable binding} (or \LPNterm{variable
instantiation}) that led to success.

But that's not the end of the story. The whole point of variables
is that they can stand for,  or
unify with,  different things. And there is information about other
women in the knowledge base. We can access this information by typing
a semicolon:
\begin{LPNcodedisplay}
X = mia ;
\end{LPNcodedisplay}
Remember that ";" means \LPNemph{or}, so this query means:
\LPNemph{are there any alternatives}?  So Prolog begins working through
the knowledge base again (it remembers where it got up to last time
and starts from there) and sees that if it unifies "X" with "jody",
then the query agrees perfectly with the second entry in the knowledge
base.  So it responds:
\begin{LPNcodedisplay}
X = mia ;
X = jody
\end{LPNcodedisplay}
It's telling us that there is information about a second woman in KB4,
and (once again) it actually gives us the value that led to success.
And of course, if we press ";" a second time, Prolog returns the
answer
\begin{LPNcodedisplay}
X = mia ;
X = jody ;
X = yolanda
\end{LPNcodedisplay}


But what happens if we press ";" a \LPNemph{third} time?  Prolog responds
no. No other unifications are possible.  There are no other facts starting
with the symbol "woman".  The last four entries in the knowledge base concern
the "love" relation, and there is no way that such entries can be unified with
a query of the form "woman(X)".

Let's try a more complicated query, namely
\begin{LPNcodedisplay}
?- loves(marsellus,X), woman(X).
\end{LPNcodedisplay}
Now, remember that "," means \LPNemph{and}, so this query says:
\LPNemph{is there any individual} "X" \LPNemph{such that Marsellus
loves} "X" \LPNemph{and} "X" \LPNemph{is a woman}?  If you look at the
knowledge base you'll see that there is: Mia is a woman (fact 1) and
Marsellus loves Mia (fact 5).  And in fact, Prolog is capable of
working this out. That is, it can search through the knowledge base
and work out that if it unifies "X" with Mia, then both conjuncts of
the query are satisfied (we'll learn in the following chapter how
Prolog does this).  So Prolog returns the answer
\begin{LPNcodedisplay}
X = mia
\end{LPNcodedisplay}



The business of unifying variables with information in the knowledge
base is the heart of Prolog.  As we'll learn, there are many
interesting ideas in Prolog --- but when you get right down to it,
it's Prolog's ability to perform unification and return the values of
the variable bindings to us that is crucial.



\subsection*{Knowledge Base 5}\label{SUBSEC.L1.KB5}


Well, we've introduced variables, but so far we've only used them in
queries. But variables not only \LPNemph{can} be used in knowledge
bases, it's only when we start to do so that we can write truly
interesting programs.  Here's a simple example, the knowledge base
KB5:

\begin{LPNcodedisplay}
loves(vincent,mia).
loves(marsellus,mia).
loves(pumpkin,honey_bunny).
loves(honey_bunny,pumpkin).

jealous(X,Y):- loves(X,Z), loves(Y,Z).
\end{LPNcodedisplay}


KB5 contains four facts about the "loves" relation and one rule.
(Incidentally, the blank line between the facts and the rule has no
meaning: it's simply there to increase the readability. As we said
earlier, Prolog gives us a great deal of freedom in the way we format
knowledge bases.)  But this rule is by far the most interesting one we
have seen so far: it contains three variables (note that "X", "Y", and
"Z" are all upper-case letters). What does it say?

In effect, it is defining a concept of jealousy.  It says that an
individual "X" will be jealous of an individual "Y" if there is some
individual "Z" that "X" loves, and "Y" loves that same individual "Z"
too. (Ok, so jealousy isn't as straightforward as this in the real
world.) The key thing to note is that this is a \LPNemph{general}
statement: it is not stated in terms of "mia", or "pumpkin", or anyone
in particular --- it's a conditional statement about
\LPNemph{everybody} in our little world.

Suppose we pose the query:
%
\begin{LPNcodedisplay}
?- jealous(marsellus,W).
\end{LPNcodedisplay}
%
This query asks: can you find an individual "W" such that Marsellus is
jealous of "W"?  Vincent is such an individual.  If you check the
definition of jealousy, you'll see that Marsellus must be jealous of
Vincent, because they both love the same woman, namely Mia. So Prolog
will return the value
%
\begin{LPNcodedisplay}
W = vincent
\end{LPNcodedisplay}


Now some questions for \LPNemph{you}. First, are there any other
jealous people in KB5? Furthermore, suppose we wanted Prolog to tell
us about all the jealous people: what query would we pose? Do any of
the answers surprise you? Do any seem silly?



\section{Prolog Syntax}\label{SEC.L1.SYNTAX}

Now that we've got some idea of what Prolog does, it's time to go back
to the beginning and work through the details more carefully. Let's
start by asking a very basic question: we've seen all kinds of
expressions (for example "jody", "playsAirGuitar(mia)", and "X") in
our Prolog programs, but these have just been examples.  It's time for
precision: exactly what are facts, rules, and queries built out of?

The answer is \LPNterm{terms}, and there are four kinds of term in
Prolog: atoms, numbers, variables, and complex terms (or structures).
Atoms and numbers are lumped together under the heading
\LPNterm{constants}, and constants and variables together make up the
\LPNterm{simple terms} of Prolog.

Let's take a closer look. To make things crystal clear, let's first be
precise about the basic \LPNterm{characters} (that is, symbols) at our
disposal.  The \LPNemph{upper-case letters} are "A", "B",\ldots,"Z";
the \LPNemph{lower-case letters} are "a", "b",\ldots,"z"; the
\LPNemph{digits} are "0", "1", "2",\ldots,"9". In addition we have the "_"
symbol, which is called \LPNterm{underscore}, and some \LPNemph{special
  characters}, which include characters such as "+", "-", "*", "/",
"<", ">", "=", ":", ".", "\&", "~".  The blank \LPNemph{space} is also
a character, but a rather unusual one, being invisible. A
\LPNterm{string} is an unbroken sequence of characters.



\subsection*{Atoms}\label{SUBSEC.L1.ATOMS}

An \LPNterm{atom} is either:

\begin{enumerate}
\item{}A string of characters made up of upper-case letters,
  lower-case letters, digits, and the underscore character, that
  begins with a lower-case letter.  Here are some examples: "butch",
  "big_kahuna_burger", "listens2Music" and "playsAirGuitar".
\item{}An arbitrary sequence of characters enclosed in single quotes.
  For example '"Vincent"', '"The Gimp"', '"Five_Dollar_Shake"',
  '"\&^%\&#@$ \&*"', and '" "'.
  The sequence of characters between the single quotes is called the \LPNterm{atom name}.
  Note that we are allowed to use spaces in such
  atoms; in fact, a common reason for using single quotes is so we
  can do precisely that.
\item{}A string of special characters.  Here are some examples: "@=" and
  "====>" and ";" and ":-" are all atoms.  As we have seen, some of
  these atoms, such as ";" and ":-" have a pre-defined meaning.
\end{enumerate}


\subsection*{Numbers}\label{SUBSEC.L1.NUMBERS}


Real \LPNterm{numbers} aren't particularly important in typical Prolog
applications.  So although most Prolog implementations do support
\LPNterm{floating point} numbers or \LPNterm{floats} (that is,
representations of real numbers such as 1657.3087 or $\pi$) we say
little about them in this book.

But \LPNterm{integers} (that is: \ldots,-2, -1, 0, 1, 2, 3,\ldots) are
useful for such tasks as counting the elements of a list, and we'll
discuss how to manipulate them in Chapter~\ref{CHAPTER5}. Their Prolog
syntax is the obvious one: "23", "1001", "0", "-365", and so on.


\subsection*{Variables}\label{SUBSEC.L1.VARS}


A \LPNterm{variable} is a string of upper-case letters, lower-case letters,
digits and underscore characters that starts \LPNemph{either} with an
upper-case letter \LPNemph{or} with an underscore.  For example, "X",
"Y", "Variable", "_tag", "X_526",  "List", "List24", "_head",
"Tail", "_input" and "Output" are all Prolog variables.

The variable "_" (that is, a single underscore character) is rather
special.  It's called the \LPNemph{anonymous variable}, and we discuss
it in Chapter~\ref{CHAPTER4}.



\subsection*{Complex terms}\label{SUBSEC.L1.COMPLEX.TERMS}

Constants, numbers, and variables are the building blocks: now we need
to know how to fit them together to make \LPNterm{complex terms}.
Recall that complex terms are often called \LPNterm{structures}.

Complex terms are built out of a \LPNterm{functor} followed by a
sequence of \LPNterm{arguments}. The arguments are put in ordinary
parentheses, separated by commas, and placed after the functor. Note
that the functor has to be directly followed by the parenthesis; you
can't have a space between the functor and the parenthesis enclosing
the arguments. The functor \LPNemph{must} be an atom.  That is,
variables \LPNemph{cannot} be used as functors.  On the other hand,
arguments can be any kind of term.

Now, we've already seen lots of examples of complex terms when we
looked at the knowledge bases KB1 to KB5. For example,
"playsAirGuitar(jody)" is a complex term: its functor is
"playsAirGuitar" and its argument is "jody".  Other examples are
"loves(vincent,mia)" and, to give an example containing a variable,
"jealous(marsellus,W)".

But  the definition allows for  more complex terms than this.
In fact, it allows us to keep nesting complex terms inside complex
terms indefinitely (that is, it is allows  \LPNterm{recursive structure}).
For example
%
\begin{LPNcodedisplay}
hide(X,father(father(father(butch))))
\end{LPNcodedisplay}
%
is a perfectly acceptable complex term. Its functor is "hide", and it
has two arguments: the variable "X", and the complex term
"father(father(father(butch)))". This complex term has "father" as its
functor, and another complex term, namely "father(father(butch))", as
its sole argument. And the argument of this complex term, namely
"father(butch)", is also complex.  But then the nesting bottoms
out, for the argument here is the constant "butch".


As we shall see, such nested (or recursively structured) terms enable
us to represent many problems naturally.  In fact the interplay
between recursive term structure and variable unification is the source
of much of Prolog's power.


The number of arguments that a complex term has is called its
\LPNterm{arity}.  For example, "woman(mia)" is a complex term of
arity 1, and "loves(vincent,mia)" is a complex term of arity 2.


Arity is important to Prolog.  Prolog would be quite happy for us to
define two predicates with the same functor but with a different
number of arguments.  For example, we are free to define a knowledge
base that defines a two-place predicate "love" (this might contain
such facts as "love(vincent,mia)"), and also a three-place "love"
predicate (which might contain such facts as
"love(vincent,marsellus,mia)"). However, if we did this, Prolog would
treat the two-place "love" and the three-place "love" as different
predicates. Later in the book (for example, when we introduce
accumulators in Chapter~5) we shall see that it can be useful to
define two predicates with the same functor but different arity.

When we need to talk about predicates and how we intend to use them
(for example, in documentation) it is usual to use a suffix "/"
followed by a number to indicate the predicate's arity.  To return to
KB2, instead of saying that it defines predicates
%
\begin{LPNcodedisplay}
listens2Music
happy
playsAirGuitar
\end{LPNcodedisplay}
%
we should really say that it defines predicates
%
\begin{LPNcodedisplay}
listens2Music/1
happy/1
playsAirGuitar/1
\end{LPNcodedisplay}
And Prolog can't get confused about a knowledge base containing the
two different love predicates, for it regards the "love/2" predicate
and the "love/3" predicate as distinct.



\section{Exercises}\label{SEC.L1.EXERCISES}

\begin{LPNexercise}{L1.EX1}Which of the following sequences of
  characters are atoms, which are variables, and which are neither?

\begin{enumerate}
\item{}"vINCENT"
\item{}"Footmassage"
\item{}"variable23"
\item{}"Variable2000"
\item{}"big_kahuna_burger"
\item{}"'big kahuna burger'"
\item{}"big kahuna burger"
\item{}"'Jules'"
\item{}"_Jules"
\item{}"'_Jules'"
\end{enumerate}
\end{LPNexercise}

\begin{LPNexercise}{L1.EX2}Which of the following sequences of
  characters are atoms, which
  are variables, which are complex terms, and which are not terms at
  all?  Give the functor and arity of each complex term.

\begin{enumerate}
\item{}"loves(Vincent,mia)"
\item{}"'loves(Vincent,mia)'"
\item{}"Butch(boxer)"
\item{}"boxer(Butch)"
\item{}"and(big(burger),kahuna(burger))"
\item{}"and(big(X),kahuna(X))"
\item{}"_and(big(X),kahuna(X))"
\item{}"(Butch kills Vincent)"
\item{}"kills(Butch Vincent)"
\item{}"kills(Butch,Vincent"
\end{enumerate}
\end{LPNexercise}

\begin{LPNexercise}{L1.EX3}How many facts, rules, clauses, and predicates are there in the
following knowledge base?  What are the heads of the rules, and what
are the goals they contain?

\begin{LPNcodedisplay}
woman(vincent).
woman(mia).
man(jules).
person(X):- man(X); woman(X).
loves(X,Y):- father(X,Y).
father(Y,Z):- man(Y), son(Z,Y).
father(Y,Z):- man(Y), daughter(Z,Y).
\end{LPNcodedisplay}
\end{LPNexercise}

\begin{LPNexercise}{L1.EX4}Represent the following in Prolog:

\begin{enumerate}
\item{}Butch is a killer.
\item{}Mia and Marsellus are married.
\item{}Zed is dead.
\item{}Marsellus kills everyone who gives Mia a footmassage.
\item{}Mia loves everyone who is a good dancer.
\item{}Jules eats anything that is nutritious or tasty.
\end{enumerate}
\end{LPNexercise}

\begin{LPNexercise}{L1.EX5}Suppose we are working with the following knowledge base:

\begin{LPNcodedisplay}
wizard(ron).
wizard(X):- hasBroom(X), hasWand(X).
hasWand(harry).
quidditchPlayer(harry).
hasBroom(X):- quidditchPlayer(X).
\end{LPNcodedisplay}


How does Prolog respond to the following queries?
\begin{enumerate}
\item{}"wizard(ron)."
\item{}"witch(ron)."
\item{}"wizard(hermione)."
\item{}"witch(hermione)."
\item{}"wizard(harry)."
\item{}"wizard(Y)."
\item{}"witch(Y)."
\end{enumerate}
\end{LPNexercise}



\section{Practical Session}\label{SEC.L1.PRAXIS}

Don't be fooled by the fact that the description of the practical
sessions is shorter than the text you have just read; the practical
part is definitely the most important. Yes, you need to read the text
and do the exercises, but that's not enough to become a Prolog
programmer. To really master the language you need to sit down in
front of a computer and play with Prolog --- a lot!

The goal of the first practical session is for you to become familiar
with the basics of how to create and run simple Prolog programs.  Now,
because there are many different implementations of Prolog, and
different operating systems you can run them under, we can't be too
specific here. Rather, what we'll do is describe in very general terms
what is involved in running Prolog, list the practical skills you
need to master, and suggest some things for you to do.

The simplest way to run a Prolog program is as follows. You have a
file with your Prolog program in it (for example, you may have a file
"kb2.pl" which contains the knowledge base KB2). You then start Prolog.
Prolog will display its prompt, something like
%
\begin{LPNcodedisplay}
?-
\end{LPNcodedisplay}
%
which indicates that it is ready to accept a query.

Now, at this stage, Prolog knows absolutely nothing about KB2 (or
indeed anything else). To see this, type in the command "listing",
\index{PROLOG listing/0@\texttt{listing/0}} followed by a full stop,
and hit return. That is, type
%
\begin{LPNcodedisplay}
?- listing.
\end{LPNcodedisplay}
%
and press the return key.

%\pagebreak  NEEDED TO GET MARGINPAR ON LEFT SIDE

Now, the \LPNterm{listing} command is a special built-in Prolog
predicate that instructs Prolog to display the contents of the current
knowledge base.  But we haven't yet told Prolog about any knowledge
bases, so it will just say
%
\begin{LPNcodedisplay}
yes
\end{LPNcodedisplay}
%
This is a correct answer: as yet Prolog knows nothing --- so it
correctly displays all this nothing and says "yes". Actually, with
more sophisticated Prolog implementations you may get a little more
(for example, the names of libraries that have been loaded; libraries
are discussed in Chapter~\ref{CHAPTER12})  but, one way or another,
you will receive what is essentially an ``I know nothing about any
knowledge bases!'' answer.

So let's tell Prolog about KB2. Assuming that you've stored KB2 in the
file "kb2.pl", and that this file is in the  directory where
you're running Prolog, all you have to type is
%
\begin{LPNcodedisplay}
?- [kb2].
\end{LPNcodedisplay}
This tells Prolog to \LPNterm{consult} the file "kb2.pl", and load the
contents as its new knowledge base. Assuming that  "kb2.pl"
contains no typos, Prolog will read it in, maybe print out a message
saying that it is consulting this file, and then answer:
%
\begin{LPNcodedisplay}
yes
\end{LPNcodedisplay}


Incidentally, it is common to store Prolog code in files with a
".pl" suffix. It's an indication of what the file contains (namely
Prolog code) and with some Prolog implementations you don't actually
have to type in the ".pl" suffix when you consult a file.  Nice ---
but there is a drawback. Files containing Perl scripts usually have a
".pl" suffix too, and nowadays there are a lot of Perl scripts in use,
so this can cause confusion. C'est la vie.

If the above doesn't work, that is, if typing
\begin{LPNcodedisplay}
?- [kb2].
\end{LPNcodedisplay}
produces an error message saying that the file "kb2"
%(also sometimes
%called "source_sink")
does not exist, then you probably haven't started Prolog from the
directory where "kb2.pl" is stored. In that case, you can either stop
Prolog (by typing "halt." after the prompt), change to the directory
where "kb2.pl" is stored, and start Prolog again. Or you can tell
Prolog exactly where to look for "kb2.pl". To do this, instead of
writing only "kb2" between the square brackets, you give Prolog the
whole path enclosed in single quotes. For example, you type something
like
\begin{LPNcodedisplay}
?- ['home/kris/Prolog/kb2.pl'].
\end{LPNcodedisplay}

or
\begin{LPNcodedisplay}
?- ['c:/Documents and Settings/Kris/Prolog/kb2.pl'].
\end{LPNcodedisplay}


Ok, so Prolog should now know about all the KB2 predicates. And we can
check whether it does by using the "listing" command again:
%
\begin{LPNcodedisplay}
?- listing.
\end{LPNcodedisplay}
%
If you do this, Prolog will list (something like) the following on the
screen:
%
\begin{LPNcodedisplay}
listens2Music(mia).
happy(yolanda).
playsAirGuitar(mia):-
   listens2Music(mia).
playsAirGuitar(yolanda):-
   listens2Music(yolanda).
listens2Music(yolanda):-
   happy(yolanda).

yes
\end{LPNcodedisplay}
That is, it will list the facts and rules that make up KB2, and then
say "yes". Once again, you may get a little more than this, such as
the locations of various libraries that have been loaded.

Incidentally, "listing" can be used in other ways. For example, typing
%
\begin{LPNcodedisplay}
?- listing(playsAirGuitar).
\end{LPNcodedisplay}
%
simply lists all the information in the knowledge base about the
"playsAirGuitar" predicate. So in this case Prolog will display
%
\begin{LPNcodedisplay}
playsAirGuitar(mia):-
   listens2Music(mia).
playsAirGuitar(yolanda):-
   listens2Music(yolanda).

yes
\end{LPNcodedisplay}



Well --- now you're ready to go. KB2 is loaded and Prolog is running,
so you can (and should!) start making exactly the sort of inquiries we
discussed in the text.

But let's back up a little, and summarise a few of the practical
skills you will need to master to get this far:

\begin{itemize}
\item{}You will need to know some basic facts about the operating
system you are using, such as the directory structure it uses. After
all, you will need to know how to save the files containing programs
where you want them.
\item{}You will need to know how to use some sort of text editor, in
order to write and modify programs. Some Prolog implementations come
with built-in text editors, but if you already know a text editor
(such as Emacs) you can use this to write your Prolog code. Just make
sure that you save your files as simple text files (for example, if you
are working under Windows, don't save them as Word documents).
\item{}You may want to take example Prolog programs from the internet.
So make sure you know how to use a browser to find what you want, and
to store the code where you want it.
\item{}You need to know how to start your version of Prolog, and how
to consult files with it.
\end{itemize}


The sooner you pick up these skills, the better. With them out of the
way (which shouldn't take long) you can start concentrating on
mastering Prolog (which will take longer).

But assuming you have mastered these skills, what next?  Quite simply,
\LPNemph{play with Prolog!} Consult the various knowledge bases
discussed in the text, and check that the queries discussed really do
work the way we said they did.  In particular, take a look at KB5 and
make sure you understand why you get those peculiar jealousy
relations.  Try posing new queries.  Experiment with the "listing"
predicate (it's a useful tool).  Type in the knowledge base used in
Exercise~\ref{L1.EX5}, and check whether your answers are correct.
Best of all, think of some simple situation that interests you, and
create a brand-new knowledge base from scratch.

