

\chapter{Lists}\label{CHAPTER4}

\Thischapter{145}{

\noindent
This chapter has three main goals:

\begin{enumerate}
\item{}To introduce lists, an important recursive data structure
often used in Prolog programming.
\item{}To define the \texttt{member/2} predicate, a fundamental Prolog tool
for manipulating lists.
\item{} To introduce the idea of recursing down
lists.
\end{enumerate}

}

\section{Lists}\label{SEC.L4.LISTS}



As its name suggests, a list is just a plain old list of
items. Slightly more precisely, it is a finite sequence of elements.
Here are some examples of lists in Prolog:


\begin{LPNcodedisplay}
[mia, vincent, jules, yolanda]

[mia, robber(honey_bunny), X, 2, mia]

[]

[mia, [vincent, jules], [butch, girlfriend(butch)]]

[[], dead(z), [2, [b, c]], [], Z, [2, [b, c]]]
\end{LPNcodedisplay}



We can learn some important things from these examples.
\begin{enumerate}
\item{}We can specify lists in Prolog by enclosing the
\LPNterm{elements} of the list in square brackets (that is, the symbols
"[" and "]").  The elements are separated by commas.  For example, the
first list shown above, "[mia, vincent, jules, yolanda]", is a list with
four elements, namely "mia", "vincent", "jules", and "yolanda".  The
\LPNterm{length} of a list is the number of elements it has, so our
first example is a list of length four.
\item{}From "[mia,robber(honey\_bunny),X,2,mia]",
our second example,
we learn that all sorts of Prolog
objects can be elements of a list.  The first element of this list is
"mia", an atom; the second element is "robber(honey\_bunny)",
a complex term; the third element is "X", a variable; the fourth
element is "2", a number.  Moreover, we also learn that the same
item may occur more than once in the same list: for example, the fifth
element of this list is "mia", which is same as the first
element.
\item{}The third example shows that there is a special list,
        the \LPNterm{empty list}.  The empty list (as its name
        suggests) is the list that contains no elements.  What is the
        length of the empty list?  Zero, of course (for the length of a
        list is the number of members it contains, and the empty list
        contains nothing).
\item{}The fourth example teaches us something extremely important:
        lists can contain other lists as elements.  For example, the
        second element of
        \begin{LPNcodedisplay}
[mia, [vincent, jules], [butch,girlfriend(butch)]
\end{LPNcodedisplay}
 is
        "[vincent,jules]". The
        third is "[butch,girlfriend(butch)]".

        What is the length of the fourth list?  The answer is: three.
If you thought it was five (or indeed, anything else) you're not
thinking about lists in the right way.  The elements of the list are
the things between the outermost square brackets separated by commas.
So this list contains \LPNemph{three} elements: the first element is
"mia", the second element is "[vincent, jules]", and the
third element is "[butch, girlfriend(butch)]".
\item{}The last example mixes all these ideas together.  We have
        here a list which contains the empty list (in fact, it contains it
        twice), the complex term "dead(z)", two copies of the list
        "[2, [b, c]]", and the variable "Z".  Note that
        the third (and the last) elements are lists which themselves
        contain lists (namely "[b, c]").
\end{enumerate}


Now for an important point.  Any non-empty list can be thought of as
consisting of two parts: the \LPNterm{head} and the \LPNterm{tail}.  The
head is simply the first item in the list; the tail is everything
else. To put it more precisely, the tail is the list
that remains when we take the first element away; that is,
\LPNemph{the tail of a list is always a list}. For example, the head
of

\begin{LPNcodedisplay}
        [mia, vincent, jules, yolanda]
\end{LPNcodedisplay}
is "mia" and the tail is " [vincent, jules, yolanda]".
Similarly, the head of
  \begin{LPNcodedisplay}
[[], dead(z), [2, [b, c]], [], Z, [2, [b, c]]]
\end{LPNcodedisplay}
is "[]", and the
tail is "[dead(z), [2,[b,c]],[],Z,[2,[b, c]]]".
And what are the head and the tail of the list "[dead(z)]"?
Well, the head is the first element of the list, which is
"dead(z)", and the tail is the list that remains if we take
the head away, which, in this case, is the empty list "[]".

What about the empty list? It has neither a head nor a tail.
That is, the empty list has no internal structure; for Prolog,
"[]" is a special, particularly simple, list.  As we shall
learn when we start writing recursive list processing programs, this
fact plays an important role in Prolog programming.

Prolog has a special built-in operator "|" which can be used to
decompose a list into its head and tail. It is important to get to
know how to use "|", for it is a key tool for writing Prolog list
manipulation programs.

The most obvious use of "|" is to extract information from lists.  We
do this by using "|" together with unification.  For example, to get
hold of the head and tail of "[mia,vincent," "jules,yolanda]" we can
pose the following query:

\begin{LPNcodedisplay}
?- [Head|Tail] = [mia, vincent, jules, yolanda].

Head = mia
Tail = [vincent,jules,yolanda]
yes
\end{LPNcodedisplay}
That is, the head of the list has become bound to "Head" and the
tail of the list has become bound to "Tail".  Note that there is
nothing special about "Head" and "Tail", they are simply
variables.  We could just as well have posed the query:


\begin{LPNcodedisplay}
?- [X|Y] = [mia, vincent, jules, yolanda].

X = mia
Y = [vincent,jules,yolanda]
yes
\end{LPNcodedisplay}



As we mentioned above, only non-empty lists have heads and tails.  If
we try to use "|" to pull "[]" apart, Prolog will fail:

\begin{LPNcodedisplay}
?- [X|Y] = [].

no
\end{LPNcodedisplay}
That is, Prolog treats "[]" as a special list.  This observation
is extremely  important.  We'll see why later.

Let's look at some other examples.  We can extract the head and tail
of the following list just as we saw above:

\begin{LPNcodedisplay}
?- [X|Y] = [[], dead(z), [2, [b, c]], [], Z].

X = []
Y = [dead(z),[2,[b,c]],[],_7800]
Z = _7800
yes
\end{LPNcodedisplay}
That is: the head of the list is bound to "X", the tail is bound
to "Y".  (We also learn that Prolog has bound
"Z" to the internal variable "\_7800".)

But we  can do a lot more with "|"; it really is a
flexible tool.  For example, suppose we wanted to know what the first
\LPNemph{two} elements of the list were, and also the remainder of the
list after the second element.  Then we'd pose the following query:

\begin{LPNcodedisplay}
?- [X,Y | W] = [[], dead(z), [2, [b, c]], [], Z].

X = []
Y = dead(z)
W = [[2,[b,c]],[],_8327]
Z = _8327
yes
\end{LPNcodedisplay}

That is, the head of the list is bound to "X", the second element is
bound to "Y", and the remainder of the list after the second element
is bound to "W" (that is, "W" is the list that remains when we take
away the first two elements). So "|" can not only be used to split a
list into its head and its tail, we can also use it to split a list at
any point. To the left of "|" we simply indicate how many elements we
want to take away from the front of the list, and then to right of the
"|" we will get what remains.

This is a good time to introduce the \LPNterm{anonymous variable}.
Suppose we were interested in getting hold of the second and fourth
elements of the list:
\begin{LPNcodedisplay}
[[], dead(z), [2, [b, c]], [], Z].
\end{LPNcodedisplay}

Now, we could find out like this:
\begin{LPNcodedisplay}
?- [X1,X2,X3,X4 | Tail] =
            [[], dead(z), [2, [b, c]], [], Z].

X1 = []
X2 = dead(z)
X3 = [2,[b,c]]
X4 = []
Tail = [_8910]
Z = _8910
yes
\end{LPNcodedisplay}

Ok, we have got the information we wanted: the values we are
interested in are bound to the variables "X2" and "X4".  But
we've got a lot of other information too (namely the values bound to
"X1", "X3" and "Tail"). And perhaps we're not
interested in all this other stuff. If so, it's a bit silly having to
explicitly introduce variables "X1", "X3" and "Tail" to
deal with it.  And in fact, there is a simpler way to obtain \LPNemph{only}
the information we want: we can pose the following query
instead:


\begin{LPNcodedisplay}
?- [_,X,_,Y|_] = [[], dead(z), [2, [b, c]], [], Z].

X = dead(z)
Y = []
Z = _9593
yes
\end{LPNcodedisplay}


The "\_" symbol (that is, underscore) is the anonymous variable.  We
use it when we need to use a variable, but we're not interested in
what Prolog instantiates the variable to.  As you can see in the above
example, Prolog didn't bother telling us what "\_" was bound to.
Moreover, note that each occurrence of "\_" is \LPNemph{independent}:
each is bound to something different. This couldn't happen with an
ordinary variable of course, but then the anonymous variable isn't
meant to be ordinary. It's simply a way of telling Prolog to bind
something to a given position, completely independently of any other
bindings.

Let's look at one last example. The third element of our working
example is a list (namely "[2, [b, c]]").  Suppose we
wanted to extract the tail of this internal list, and that we are not
interested in any other information.  How could we do this?  As
follows:

\begin{LPNcodedisplay}
?- [_,_,[_|X]|_] =
      [[], dead(z), [2, [b, c]], [], Z, [2, [b, c]]].

X = [[b,c]]
Z = _10087
yes
\end{LPNcodedisplay}




\section{Member}\label{SEC.L4.MEMBER}

It's time to look at our first example of a recursive Prolog program
for manipulating lists.  One of the most basic things we would like to
know is whether something is an element of a list or not.  So let's
write a program that, when given as inputs an arbitrary object
\LPNemph{X} and a list \LPNemph{L}, tells us whether or not
\LPNemph{X} belongs to \LPNemph{L}.  The program that does this is
usually called \LPNterm{member}\index{PROLOG
member/2@\texttt{member/2}}, and it is the simplest example of a
Prolog program that exploits the recursive structure of lists.  Here
it is:


\begin{LPNcodedisplay}
member(X,[X|T]).
member(X,[H|T]) :- member(X,T).
\end{LPNcodedisplay}


That's all there is to it: one fact (namely "member(X,[X|T])")
and one rule (namely "member(X,[H|T]) :- member(X,T)"). But note
that the rule is recursive (after all, the functor "member"
occurs in both the rule's head and body) and it is this that explains
why such a short program is all that is required.
Let's take a closer look.

We'll start by reading the program declaratively. And read this way,
it is obviously sensible.  The first clause (the fact) simply says: an
object "X" is a member of a list if it is the head of that list.
Note that we used the built-in "|" operator to state this (simple
but important) principle about lists.

What about the second clause, the recursive rule?  This says: an
object "X" is member of a list if it is a member of the tail of
the list.  Again, note that we used the "|" operator to state
this principle.

Now, clearly this definition makes good declarative sense.  But does
this program actually \LPNemph{do} what it is supposed to do?  That is,
will it really tell us whether an object "X" belongs to a list
"L"? And if so, how exactly does it do this?  To answer such
questions, we need to think about its procedural meaning.  Let's work
our way through a few examples.

Suppose we posed the following query:

\begin{LPNcodedisplay}
?- member(yolanda,[yolanda,trudy,vincent,jules]).
\end{LPNcodedisplay}
Prolog will immediately answer yes.  Why?  Because it can unify
"yolanda" with both occurrences of "X" in the first clause
(the fact) in the definition of "member/2", so it succeeds
immediately.

Next consider the following query:

\begin{LPNcodedisplay}
?- member(vincent,[yolanda,trudy,vincent,jules]).
\end{LPNcodedisplay}
Now the first rule won't help ("vincent" and "yolanda" are
distinct atoms) so Prolog goes to the second clause, the recursive
rule. This gives Prolog a new goal: it now has to see if

\begin{LPNcodedisplay}
?- member(vincent,[trudy,vincent,jules]).
\end{LPNcodedisplay}
Once again the first clause won't help, so Prolog goes (again) to
the recursive rule. This gives it a new goal, namely

\begin{LPNcodedisplay}
?- member(vincent,[vincent,jules]).
\end{LPNcodedisplay}
This time, the first clause does help, and the query succeeds.

So far so good, but we need to ask an important question.  What
happens when we pose a query that \LPNemph{fails}?  For example, what
happens if we pose the query

\begin{LPNcodedisplay}
?- member(zed,[yolanda,trudy,vincent,jules]).
\end{LPNcodedisplay}


Now, this should obviously fail (after all, "zed" is not on the
list).  So how does Prolog handle this?  In particular, how can we be
sure that Prolog really will \LPNemph{stop}, and say \LPNemph{no}, instead
going into an endless recursive loop?

Let's think this through systematically.  Once again, the first clause
cannot help, so Prolog uses the recursive rule, which gives it a new
goal

\begin{LPNcodedisplay}
?- member(zed,[trudy,vincent,jules]).
\end{LPNcodedisplay}
Again, the first clause doesn't help, so Prolog reuses the recursive
rule and tries to show that

\begin{LPNcodedisplay}
?- member(zed,[vincent,jules]).
\end{LPNcodedisplay}
Similarly, the first rule doesn't help, so Prolog reuses the second
rule yet again and tries the goal
\begin{LPNcodedisplay}
?- member(zed,[jules]).
\end{LPNcodedisplay}
Again the first clause doesn't help, so Prolog uses the second rule,
which gives it the goal

\begin{LPNcodedisplay}
?- member(zed,[])
\end{LPNcodedisplay}
And \LPNemph{this} is where things get interesting.  Obviously the
first clause can't help here.  But note: \LPNemph{the recursive rule
can't do anything more either}.  Why not? Simple: the recursive rule
relies on splitting the list into a head and a tail, but as we have
already seen, the empty list \LPNemph{can't} be split up in this way.
So the recursive rule cannot be applied either, and Prolog stops
searching for more solutions and announces no.  That is, it tells us
that "zed" does not belong to the list, which is just what it ought to
do.

We could summarise the "member/2" predicate as follows.  It
is a recursive predicate, which systematically searches down the
length of the list for the required item.  It does this by stepwise
breaking down the list into smaller lists, and looking at the first
item of each smaller list.  This mechanism that drives this
search is recursion, and the reason that this recursion is safe (that
is, the reason it does not go on forever) is that at the end of the
line Prolog has to ask a question about the empty list.  The empty
list \LPNemph{cannot} be broken down into smaller parts, and this allows a
way out of the recursion.

Well, we've now seen why "member/2" works, but in fact it's
far more useful than the previous example might suggest.  Up till now
we've only been using it to answer yes/no questions.  But we can also
pose questions containing variables.  For example, we can have the
following dialog with Prolog:


\begin{LPNcodedisplay}
member(X,[yolanda,trudy,vincent,jules]).

X = yolanda ;

X = trudy ;

X = vincent ;

X = jules ;

no
\end{LPNcodedisplay}


That is, Prolog has told us what every member of a list is.  This is
an extremely common use of "member/2". In effect, by using the
variable we are saying to Prolog: ``Quick! Give me some element of the
list!''. In many applications we need to be able to extract members of
a list, and this is the way it is typically done.



One final remark.  The way we defined "member/2" above is
certainly correct, but in one respect it is a little messy.

Think about it. The first clause is there to deal with the head of the
list. But although the tail is irrelevant to the first clause, we
named the tail using the variable "T". Similarly,
the recursive rule is there to deal with the tail of the list.  But
although the head is irrelevant here, we named it
using the variable  "H". These unnecessary variable
names are distracting: it's better to write predicates in a way that
focuses attention on what is really important in each clause, and the
anonymous variable gives us a nice way of doing this. That is, we  can rewrite
"member/2" as follows:
\begin{LPNcodedisplay}
member(X,[X|_]).
member(X,[_|T]) :- member(X,T).
\end{LPNcodedisplay}

This version is exactly the same, both declaratively and procedurally.
But it's just that little bit clearer: when you read it, you are
forced to concentrate on what is essential.


\section{Recursing down Lists}\label{SEC.L4.RDAL}

The "member/2" predicate works by recursively working its way down a
list, doing something to the head, and then recursively doing the same
thing to the tail. Recursing down a list (or indeed, several lists) in
this way is extremely common in Prolog; so common, in fact, that it is
important that you really master the technique. So let's look at another
example.

When working with lists, we often want to compare one list
with another, or to copy bits of one list into another, or to
translate the contents of one list into another, or something similar.
Here's an example. Let's suppose we need a predicate
"a2b/2" that takes two lists as arguments, and succeeds if
the first argument is a list of "a"s, and the second argument is
a list of "b"s of exactly the same length.  For example, if we
pose the following query
\begin{LPNcodedisplay}
?- a2b([a,a,a,a],[b,b,b,b]).
\end{LPNcodedisplay}
we want Prolog to say yes. On the other hand, if we pose the query
\begin{LPNcodedisplay}
?- a2b([a,a,a,a],[b,b,b]).
\end{LPNcodedisplay}
or the query
\begin{LPNcodedisplay}
?- a2b([a,c,a,a],[b,b,5,4]).
\end{LPNcodedisplay}
we want Prolog to say no.

When faced with such tasks, often the best way to set about solving
them is to start by thinking about the simplest possible case. Now,
when working with lists, thinking about the simplest case often means
thinking about the empty list, and it certainly means this here.
After all: what is the shortest possible list of "a"s?  It's the empty
list. Why? Because it contains no "a"s at all. And what is the
shortest possible list of "b"s? Again, the empty list: no "b"s
whatsoever in that.  So the most basic information our definition
needs to contain is
\begin{LPNcodedisplay}
?- a2b([],[]).
\end{LPNcodedisplay}
This records the obvious fact that the empty list contains exactly as
many "a"s as "b"s. But although obvious, this fact turns out
to play an important role in our program, as we shall see.

 So far so good: but how do we proceed? Here's the idea: for longer
lists, \LPNemph{think recursively}. So: when should "a2b/2"
decide that two non-empty lists are a list of "a"s and a list of
"b"s of exactly the same length? Simple: when the head of the
first list is an "a", and the head of the second list is a
"b", and "a2b/2" decides that the two tails are lists
of "a"s and "b"s of exactly the same length!  This
immediately gives us the following rule:
\begin{LPNcodedisplay}
?- a2b([a|Ta],[b|Tb]) :- a2b(Ta,Tb).
\end{LPNcodedisplay}
This says: the "a2b/2" predicate should succeed if its
first argument is a list with head "a", its second argument is a
list with head "b", and "a2b/2" succeeds on the two
tails.

Now, this definition make good sense declaratively. It is a simple and
natural recursive predicate, the base clause dealing with the empty
list, the recursive clause dealing with non-empty lists.  But how does
it work in practice? That is, what is its procedural meaning?  For
example, if we pose the query
\begin{LPNcodedisplay}
?- a2b([a,a,a],[b,b,b]).
\end{LPNcodedisplay}
Prolog will say yes, which is what we want --- but \LPNemph{why} exactly does
this happen?

Let's work the example through. In this query, neither list is empty,
so the fact does not help. Thus Prolog goes on to try the recursive
rule. Now, the query does match the rule (after all, the head of the
first list is "a" and the head of the second is "b") so
Prolog now has a new goal, namely
\begin{LPNcodedisplay}
?- a2b([a,a],[b,b]).
\end{LPNcodedisplay}
Once again, the fact does not help with this, but the recursive rule
can be used again, leading to the following goal:
\begin{LPNcodedisplay}
?- a2b([a],[b]).
\end{LPNcodedisplay}
Yet again the fact does not help, but the recursive rule does, so we
get the following goal:
\begin{LPNcodedisplay}
?- a2b([],[]).
\end{LPNcodedisplay}
At last we can use the fact: this tells us that, yes, we really do
have two lists here that contain exactly the same number of "a"s
and "b"s (namely, none at all). And because this goal succeeds,
this means that the goal
\begin{LPNcodedisplay}
?- a2b([a],[b]).
\end{LPNcodedisplay}
succeeds too. This in turn means that the goal
\begin{LPNcodedisplay}
?- a2b([a,a],[b,b]).
\end{LPNcodedisplay}
succeeds, and thus that the original goal
\begin{LPNcodedisplay}
?- a2b([a,a,a],[b,b,b]).
\end{LPNcodedisplay}
is satisfied.

We could summarise this process as follows.  Prolog started with two
lists. It peeled the head off each of them, and checked that they were
an "a" and a "b", respectively, as required. It then recursively
analysed the tails of both lists. That is, it worked its way down both
tails simultaneously, checking that at each stage the tails were
headed by an "a" and a "b".  Why did the process stop?  Because at
each recursive step we had to work with shorter lists (namely the
tails of the lists examined at the previous step) and eventually we
ended up with empty lists.  At this point, our rather trivial looking
fact was able to play a vital role: it said yes. This halted the
recursion, and ensured that the original query succeeded.

It's is also important to think about what  happens with queries
that \LPNemph{fail}. For example, if we pose the query
\begin{LPNcodedisplay}
?- a2b([a,a,a,a],[b,b,b]).
\end{LPNcodedisplay}
Prolog will correctly say no. Why? because after carrying out the
peel-off-the-head-and-recursively-examine-the-tail process three
times, it will be left with the query
\begin{LPNcodedisplay}
?- a2b([a],[]).
\end{LPNcodedisplay}
But this goal cannot be satisfied.
And if we pose the query
\begin{LPNcodedisplay}
?- a2b([a,c,a,a],[b,b,5,4]).
\end{LPNcodedisplay}
after carrying out the
peel-off-the-head-and-recursively-examine-the-tail process once,
Prolog will have the goal
\begin{LPNcodedisplay}
?- a2b([c,a,a],[b,5,4]).
\end{LPNcodedisplay}
and again, this cannot be satisfied.

Well, that's how "a2b/2" works in simple cases, but we
haven't exhausted its possibilities yet. As always with Prolog, it's a
good idea to investigate what happens when variables as used as
input. And with "a2b/2" something interesting happens: it
acts as a translator, translating lists of "a"s to lists of
"b"s, and vice versa.  For example the query
\begin{LPNcodedisplay}
?- a2b([a,a,a,a],X).
\end{LPNcodedisplay}
yields the response
\begin{LPNcodedisplay}
X = [b,b,b,b].
\end{LPNcodedisplay}
That is, the list of "a"s has been translated to a list of
"b"s. Similarly, by using a variable in the first argument
position, we can use it to translate lists of "b"s
to lists of "a"s:
\begin{LPNcodedisplay}
?- a2b(X,[b,b,b,b]).

?- X = [a,a,a,a]
\end{LPNcodedisplay}

And of course, we can use variables in both argument positions:
\begin{LPNcodedisplay}
?- a2b(X,Y).
\end{LPNcodedisplay}
Can you work out what happens in this case?

To sum up: "a2b/2" is an extremely simple example of a
program that works by recursing its way down a pair of lists. But
don't be fooled by its simplicity: the kind of programming it
illustrates is fundamental to Prolog.  Both its declarative form (a
base clause dealing with the empty list, a recursive clause dealing
with non-empty lists) and the procedural idea it trades on (do
something to the heads, and then recursively do the same thing to the
tails) come up again and again in Prolog programming. In fact, in the
course of your Prolog career, you'll find that you'll write what is
essentially the "a2b/2" predicate, or a more complex
variant of it, many times over in many different guises.

\section{Exercises}\label{SEC.L4.EXERCISES}

\begin{LPNexercise}{L4.EX1}How does Prolog respond to the following queries?
\begin{enumerate}
\item{}"[a,b,c,d] = [a,[b,c,d]]."
\item{}"[a,b,c,d] = [a|[b,c,d]]."
\item{}"[a,b,c,d] = [a,b,[c,d]]."
\item{}"[a,b,c,d] = [a,b|[c,d]]."
\item{}"[a,b,c,d] = [a,b,c,[d]]."
\item{}"[a,b,c,d] = [a,b,c|[d]]."
\item{}"[a,b,c,d] = [a,b,c,d,[]]."
\item{}"[a,b,c,d] = [a,b,c,d|[]]."
\item{}"[] = \_."
\item{}"[] = [\_]."
\item{}"[] = [\_|[]]."
\end{enumerate}
\end{LPNexercise}
\begin{LPNexercise}{L4.EX1.2}Which of the following are syntactically correct lists? If the representation is correct, how many elements does the list have?

\begin{enumerate}
\item{}"[1|[2,3,4]]"
\item{}"[1,2,3|[]]"
\item{}"[1|2,3,4]"


\item{}"[1|[2|[3|[4]]]]"
\item{}"[1,2,3,4|[]]"
\item{}"[[]|[]]"
\item{}"[[1,2]|4]"
\item{}"[[1,2],[3,4]|[5,6,7]]"


\end{enumerate}
\end{LPNexercise}


\begin{LPNexercise}{L6.EX3a}
Write a predicate "second(X,List)" which checks
whether "X" is the second element of "List".
\end{LPNexercise}


\begin{LPNexercise}{L6.EX3b}
Write a predicate "swap12(List1,List2)" which checks
whether "List1" is identical to "List2", except that the
first two elements are exchanged.
\end{LPNexercise}


\begin{LPNexercise}{L4.EX2}Suppose we are given a knowledge base with the following facts:

\begin{LPNcodedisplay}
tran(eins,one).
tran(zwei,two).
tran(drei,three).
tran(vier,four).
tran(fuenf,five).
tran(sechs,six).
tran(sieben,seven).
tran(acht,eight).
tran(neun,nine).
\end{LPNcodedisplay}



Write a predicate "listtran(G,E)" which translates a list of
German number words to the corresponding list of English number words.
For example:
\begin{LPNcodedisplay}
listtran([eins,neun,zwei],X).
\end{LPNcodedisplay}
should give:
\begin{LPNcodedisplay}
X = [one,nine,two].
\end{LPNcodedisplay}

Your program should also work in the other direction. For example, if
you give it the query
\begin{LPNcodedisplay}
listtran(X,[one,seven,six,two]).
\end{LPNcodedisplay}
it should return:
\begin{LPNcodedisplay}
X = [eins,sieben,sechs,zwei].
\end{LPNcodedisplay}


(Hint: to answer this question, first ask yourself ``How do I
translate the \LPNemph{empty} list of number words?''.  That's the base
case.  For non-empty lists, first translate the head of the list, then
use recursion to translate the tail.)
\end{LPNexercise}


\begin{LPNexercise}{L4.EX3}Write a predicate "twice(In,Out)"
whose left argument is a list,
and whose right argument is a list consisting of every element in the
left list written twice.  For example, the query
\begin{LPNcodedisplay}
twice([a,4,buggle],X).
\end{LPNcodedisplay}
should return
\begin{LPNcodedisplay}
X = [a,a,4,4,buggle,buggle]).
\end{LPNcodedisplay}
And the query
\begin{LPNcodedisplay}
twice([1,2,1,1],X).
\end{LPNcodedisplay}
should return
\begin{LPNcodedisplay}
X = [1,1,2,2,1,1,1,1].
\end{LPNcodedisplay}

(Hint: to answer this question, first ask yourself ``What should happen
when the first argument is the \LPNemph{empty} list?''.  That's the base case.
For non-empty lists, think about what you should do with the head, and
use recursion to handle the tail.)
\end{LPNexercise}


\begin{LPNexercise}{L4.EX4}Draw the search trees for the following three queries:

\begin{LPNcodedisplay}
?- member(a,[c,b,a,y]).

?- member(x,[a,b,c]).

?- member(X,[a,b,c]).

\end{LPNcodedisplay}
(Search trees were introduced in Chapter~\ref{CHAPTER2}.)
\end{LPNexercise}


\section{Practical Session}\label{SEC.L4.PRAXIS}



The purpose of Practical Session 4 is to help you get familiar with
the idea of recursing down lists.  We first suggest some traces for
you to carry out, and then some programming exercises.

First, systematically carry out a number of traces on
"a2b/2" to make sure you fully understand how it
works. In particular:

\begin{enumerate}
\item{}Trace some examples, not involving variables, that succeed. For
example, trace the query "a2b([a,a,a,a],[b,b,b,b])" and relate the
output to the discussion in the text.
\item{}Trace some simple examples that fail. Try
examples involving  lists of different lengths
(such as "a2b([a,a,a,a],[b,b,b])") and
examples involving  symbols other than "a" and "b"
(such as "a2b([a,c,a,a],[b,b,5,4])").
\item{}Trace some examples involving variables. For example, try
tracing "a2b([a,a,a,a],X)" and "a2b(X,[b,b,b,b])".
\item{}Make sure you understand what happens when
both arguments in the query are variables. For example, carry
out a trace on the query "a2b(X,Y)".
\item{}Carry out a series of similar traces involving
"member/2". That is, carry out traces involving simple queries that
succeed (such as "member(a,[1,2,a,b])"), simple queries that fail
(such as "member(z,[1,2,a,b])"), and queries involving variables
(such as "member(X,[1,2,a,b])").  In all cases, make sure that
you understand why the recursion halts.
\end{enumerate}


Having done this, try the following.

\begin{enumerate}
\item{}Write a 3-place predicate "combine1" which takes three
lists as arguments and combines the elements of the first two lists
into the third as follows:
\begin{LPNcodedisplay}
?- combine1([a,b,c],[1,2,3],X).

X = [a,1,b,2,c,3]

?- combine1([f,b,yip,yup],[glu,gla,gli,glo],Result).

Result = [f,glu,b,gla,yip,gli,yup,glo]
\end{LPNcodedisplay}

\item{}Now write a 3-place predicate "combine2" which takes three
lists as arguments and combines the elements of the first two lists
into the third as follows:
\begin{LPNcodedisplay}
?- combine2([a,b,c],[1,2,3],X).

X = [[a,1],[b,2],[c,3]]

?- combine2([f,b,yip,yup],[glu,gla,gli,glo],Result).

Result = [[f,glu],[b,gla],[yip,gli],[yup,glo]]
\end{LPNcodedisplay}

\item{}Finally, write a 3-place predicate "combine3" which takes
three lists as arguments and combines the elements of the first two lists
into the third as follows:
\begin{LPNcodedisplay}
?- combine3([a,b,c],[1,2,3],X).

X = [j(a,1),j(b,2),j(c,3)]

?- combine3([f,b,yip,yup],[glu,gla,gli,glo],R).

R = [j(f,glu),j(b,gla),j(yip,gli),j(yup,glo)]
\end{LPNcodedisplay}

\end{enumerate}

All three programs are pretty much the same as "a2b/2" (though they
manipulate three lists, not two).  That is, all three can be written
by recursing down the lists, doing something to the heads, and then
recursively doing the same thing to the tails.  Indeed, once you have
written "combine1", you just need to change what you do to the heads
to get "combine2" and "combine3".

