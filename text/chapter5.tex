
\chapter{Arithmetic}\label{CHAPTER5}

\Thischapter{120}{

\noindent
This chapter has two main goals:

\begin{enumerate}
\item{}To introduce Prolog's built-in abilities for performing
       arithmetic.
\item{}To apply them to simple list processing problems, using
       accumulators.
\end{enumerate}

}

\section{Arithmetic in Prolog}\label{SEC.L5.ARITHMETIC}



Prolog provides a number of basic arithmetic tools for manipulating
integers (that is, numbers of the form ...-3, -2, -1, 0, 1, 2, 3,
4...).  Most Prolog implementation also provide tools for handling
real numbers (or floating point numbers) such as 1.53 or $6.35\times
10^{5}$, but we're not going to discuss these, for they are not
particularly useful for the symbolic processing tasks discussed in
this book.  Integers, on the other hand, are useful in connection with
symbolic tasks (we use them to state the length of lists, for example)
so it is important to understand how to work with them.  We'll start
by looking at how Prolog handles the four basic operations of
addition, multiplication, subtraction, and division.

\begin{center}\begin{tabular}{ll}
Arithmetic examples & Prolog Notation\\
$6+2=8$&"8 is 6+2."\\
$6*2=12$&"12 is 6*2."\\
$6-2=4$&"4 is 6-2."\\
$6-8=-2$&"-2 is 6-8."\\
$6\div 2=3$&"3 is 6/2."\\
$7\div 2=3$&"3 is 7/2."\\
1 is the remainder when 7 is divided by 2&"1 is mod(7,2)."
\end{tabular}\end{center}
Note that as we are working with integers, division gives us back an
integer answer.  Thus $7\div 2$ gives 3 as an answer,  leaving
remainder 1.

Posing the following queries yields the following responses\index{PROLOG is/2@\texttt{is/2}}:

\begin{LPNcodedisplay}
?- 8 is 6+2.
yes

?- 12 is 6*2.
yes

?- -2 is 6-8.
yes

?- 3 is 6/2.
yes

?- 1 is mod(7,2).
yes
\end{LPNcodedisplay}


More importantly, we can work out the answers to arithmetic questions
by using variables.  For example:

\begin{LPNcodedisplay}
?- X is 6+2.

X = 8

?- X is 6*2.

X = 12

?- R is mod(7,2).

R = 1
\end{LPNcodedisplay}

Moreover, we can use arithmetic operations when we define predicates.
Here's a simple example.  Let's define a predicate
"add_3_and_double/2" whose arguments are both integers.  This
predicate takes its first argument, adds three to it, doubles the
result, and returns the number obtained as the second argument.  We
define this predicate as follows:
\begin{LPNcodedisplay}
add_3_and_double(X,Y) :- Y is (X+3)*2.
\end{LPNcodedisplay}
And indeed, this works:
\begin{LPNcodedisplay}
?- add_3_and_double(1,X).

X = 8

?- add_3_and_double(2,X).

X = 10
\end{LPNcodedisplay}


One other thing.  Prolog understands the usual conventions we use for
disambiguating arithmetical expressions.  For example, when we write
$3+2\times 4$ we mean $3+(2\times 4)$ and not $(3+2)\times 4$, and
Prolog knows this convention:
\begin{LPNcodedisplay}
?- X is 3+2*4.

X = 11
\end{LPNcodedisplay}





\section{A Closer Look}\label{SEC.L5.LOOK}

That's the basics, but we need to know more.  The most important to
grasp is this: +, *, -, $\div$ and "mod" do \LPNemph{not} carry out
any arithmetic.  In fact, expressions such as "3+2", "3-2" and "3*2"
are simply terms.  The functors of these terms are "+", "-" and "*"
respectively, and the arguments are "3" and "2".  Apart from the fact
that the functors go between their arguments (instead of in front of
them) these are ordinary Prolog terms, and unless we do something
special, Prolog will not actually do any arithmetic.  In
particular, if we pose the query
\begin{LPNcodedisplay}
?- X = 3+2
\end{LPNcodedisplay}
we don't get back the answer "X=5".
Instead we get back
\begin{LPNcodedisplay}
X = 3+2
yes
\end{LPNcodedisplay}
That is, Prolog has simply unified the variable "X" to the complex
term "3+2".  It has \LPNemph{not} carried out any arithmetic.  It has
simply done what it usually does when \texttt{=/2} is used: performed
unification.

Similarly,
if we pose the query
\begin{LPNcodedisplay}
?- 3+2*5 = X
\end{LPNcodedisplay}
we get the response
\begin{LPNcodedisplay}
X = 3+2*5
yes
\end{LPNcodedisplay}
Again, Prolog has simply bound the variable "X" to the complex
term "3+2*5".  It did not evaluate this expression to 13.

To force Prolog to actually evaluate arithmetic expressions we have to
use
        \begin{LPNcodedisplay}
is
\end{LPNcodedisplay}
just as we did in our  earlier examples.  In fact, "is" does
something very special: it sends a signal to Prolog that says ``Hey!
Don't treat this expression as an ordinary complex term!  Call up
your built-in arithmetic capabilities and carry out the calculations!''

In short, "is" forces Prolog to act in an unusual way.  Normally
Prolog is quite happy just unifying variables to structures: that's
its job, after all.  Arithmetic is something extra that has been
bolted on to the basic Prolog engine because it is useful.
Unsurprisingly, there are some restrictions on this extra ability, and
we need to know what they are.

For a start, the arithmetic expressions to be evaluated must be on
the right hand side of "is".  In our earlier examples we carefully
posed the query
\begin{LPNcodedisplay}
?- X is 6+2.

X = 8
\end{LPNcodedisplay}
which is the right way to do it. If instead we
had asked
\begin{LPNcodedisplay}
6+2 is X.
\end{LPNcodedisplay}
we would have got a message saying "instantiation_error",
or something similar.

Moreover, although we are free to use variables on the right hand side
of "is", when we actually carry out evaluation, the variable must
already have been instantiated to a variable-free arithmetic
expression.  If the variable is uninstantiated, or if it is
instantiated to something other than an integer, we will get some sort
of "instantiation_error" message.  This is because arithmetic isn't
performed using Prolog's usual unification and knowledge base search
mechanisms: it's done by calling up a special black box which knows
about integer arithmetic.  If we hand the black box the wrong kind of
data, it's going to complain.

Here's an example. Recall our ``add 3 and double it''
predicate.
\begin{LPNcodedisplay}
add_3_and_double(X,Y) :- Y is (X+3)*2.
\end{LPNcodedisplay}
When we described this predicate, we carefully said that it added 3 to
its first argument, doubled the result, and returned the answer in its
second argument.  For example, "add_3_and_double(3,X)" returns
"X = 12".  We didn't say anything about using this predicate in
the reverse direction.  For example, we might hope that posing the
query
\begin{LPNcodedisplay}
?- add_3_and_double(X,12).
\end{LPNcodedisplay}
would return the answer "X=3".   But it doesn't. Instead we
get the "instantiation_error" message.  Why?  Well, when we pose
the query this way round, we are asking Prolog to evaluate
"12 is (X+3)*2", which it \LPNemph{can't} do as "X"
is not instantiated.

Two final remarks.  As we've already mentioned, for Prolog "3 + 2"
is just a term.  In fact, for Prolog, it really \LPNemph{is} the term
\LPNemph{+(3,2)}.  The expression "3 + 2" is just a user-friendly
notation that's nicer for us to use.  This means that, if you
want to, you can give Prolog queries like
\begin{LPNcodedisplay}
X is +(3,2)
\end{LPNcodedisplay}
and Prolog will correctly reply
\begin{LPNcodedisplay}
X = 5
\end{LPNcodedisplay}
Actually, you can even given Prolog the query
\begin{LPNcodedisplay}
?- is(X,+(3,2))
\end{LPNcodedisplay}
and Prolog will respond
\begin{LPNcodedisplay}
X = 5
\end{LPNcodedisplay}
This is because, for Prolog, the expression "X is +(3,2)" really is
the term "is(X,+(3,2))".  The expression "X is +(3,2)" is just
user-friendly notation.  Underneath, as always, Prolog is just working
away with terms.

Summing up, arithmetic in Prolog is easy to use.  Pretty much all you
have to remember is to use "is" to force evaluation, that stuff to be
evaluated must go to the right of "is", and to take care that any
variables are correctly instantiated.  But there is a deeper point
that is worth reflecting on: bolting on the extra capability to do
arithmetic in this way has further widened the gap between the
procedural and declarative meanings of Prolog programs.





\section{Arithmetic and Lists}\label{SEC.L5.ARITHMETIC-AND-LISTS}

Probably the most important use of arithmetic in this book is to tell
us useful facts about data-structures, such as lists.  For example, it
can be useful to know how long a list is.  We'll give some examples of
using lists together with arithmetic capabilities.

How long is a list? Here's a recursive definition.
\begin{enumerate}
\item{}The empty list has length zero.
\item{}A non-empty list has length
         1 + \LPNemph{len}(T), where \LPNemph{len}(T) is the length of its
        tail.
\end{enumerate}


This definition is practically a Prolog program already.  Here's the
code we need:
\begin{LPNcodedisplay}
len([],0).
len([_|T],N) :- len(T,X), N is X+1.
\end{LPNcodedisplay}

This predicate works in the expected way.
For example:
\begin{LPNcodedisplay}
?- len([a,b,c,d,e,[a,b],g],X).

X = 7
\end{LPNcodedisplay}

\clearpage
Now, this is quite a good program: it's easy to understand and
efficient.  But there is another method of finding the length of a
list.  We'll now look at this alternative, because it introduces the
idea of \LPNterm{accumu\-lators}. If you're used to other programming
languages, you're probably used to the idea of using variables to hold
intermediate results.  An accumulator is the Prolog analog of this
idea.


Here's how to use an accumulator to calculate the length of a list.
We shall define a predicate "accLen"/3 which takes the following
arguments.
\begin{LPNcodedisplay}
accLen(List,Acc,Length)
\end{LPNcodedisplay}
Here "List" is the list whose length we want to find, and "Length" is
its length (an integer).  What about "Acc"?  This is the accumulator
we will use to keep track of intermediate values for length (so it
will also be an integer).  Here's what we do.  When we call this
predicate, we are going to give "Acc" an initial value of "0".  We
then recursively work our way down the list, adding "1" to "Acc" each
time we find a head element, until we reach the empty list.  When we
reach the empty list, "Acc" will contain the length of the list.
Here's the code:
\begin{LPNcodedisplay}
accLen([_|T],A,L) :-  Anew is A+1, accLen(T,Anew,L).
accLen([],A,A).
\end{LPNcodedisplay}

The base case of the definition, unifies the second and third
arguments.  Why?  Because this trivial unification is a nice way of
making sure that the result, that is, the length of the list, is
returned. When we reach the end of the list, the accumulator (the
second variable) contains the length of the list.  So we give this
value (via unification) to the length variable (the third variable).
Here's an example trace. You can clearly see how the length variable
gets its value at the bottom of the recursion and passes it upwards as
Prolog is coming out of the recursion.
\begin{LPNcodedisplay}
?- accLen([a,b,c],0,L).
   Call: (6) accLen([a, b, c], 0, _G449) ?
   Call: (7) _G518 is 0+1 ?
   Exit: (7) 1 is 0+1 ?
   Call: (7) accLen([b, c], 1, _G449) ?
   Call: (8) _G521 is 1+1 ?
   Exit: (8) 2 is 1+1 ?
   Call: (8) accLen([c], 2, _G449) ?
   Call: (9) _G524 is 2+1 ?
   Exit: (9) 3 is 2+1 ?
   Call: (9) accLen([], 3, _G449) ?
   Exit: (9) accLen([], 3, 3) ?
   Exit: (8) accLen([c], 2, 3) ?
   Exit: (7) accLen([b, c], 1, 3) ?
   Exit: (6) accLen([a, b, c], 0, 3) ?
\end{LPNcodedisplay}

As a final step, we'll define a predicate which calls "accLen"
for us, and gives it the initial value of 0:
\begin{LPNcodedisplay}
leng(List,Length) :- accLen(List,0,Length).
\end{LPNcodedisplay}
So now we can pose queries like this:
\begin{LPNcodedisplay}
?- leng([a,b,c,d,e,[a,b],g],X).
\end{LPNcodedisplay}


Accumulators are extremely common in Prolog programs.  (We'll see
another accumulator based program in this chapter, and some more in
later chapters.) But why is this? In what way is "accLen" better than
"len"? After all, it looks more difficult. The answer is that "accLen"
is \LPNterm{tail recursive} while "len" is not. In tail recursive
programs, the result is fully calculated once we reached the bottom of
the recursion and just has to be passed up. In recursive programs
which are not tail recursive, there are goals at other levels of
recursion which have to wait for the answer from a lower level of
recursion before they can be evaluated. To understand this, compare
the traces for the queries "accLen([a,b,c],0,L)" (see above) and
"len([a,b,c],0,L)" (given below). In the first case the result is
built while going into the recursion --- once the bottom is reached at
"accLen([],3,_G449)", the result is there and only has to be passed
up. In the second case the result is built while coming out of the
recursion; the result of "len([b,c], _G481)", for instance, is only
computed after the recursive call of "len" has been completed and the
result of "len([c],_G489)" is known. In short, tail recursive programs
have less bookkeeping overhead, and this makes them more efficient.

\begin{LPNcodedisplay}
?- len([a,b,c],L).
   Call: (6) len([a, b, c], _G418) ?
   Call: (7) len([b, c], _G481) ?
   Call: (8) len([c], _G486) ?
   Call: (9) len([], _G489) ?
   Exit: (9) len([], 0) ?
   Call: (9) _G486 is 0+1 ?
   Exit: (9) 1 is 0+1 ?
   Exit: (8) len([c], 1) ?
   Call: (8) _G481 is 1+1 ?
   Exit: (8) 2 is 1+1 ?
   Exit: (7) len([b, c], 2) ?
   Call: (7) _G418 is 2+1 ?
   Exit: (7) 3 is 2+1 ?
   Exit: (6) len([a, b, c], 3) ?
\end{LPNcodedisplay}


\section{Comparing Integers}\label{SEC.L5.COMPARING-INTEGERS}



Some Prolog arithmetic predicates actually do carry out arithmetic
all by themselves (that is, without the assistance of "is").
These are the operators that compare integers.

\begin{center}\begin{tabular}{ll}
Arithmetic examples & Prolog Notation\\
$x<y$&"X < Y."\\ \index{PROLOG </2@\texttt{</2}}
$x\le y$&"X =< Y."\\ \index{PROLOG =</2@\texttt{=</2}}
$x=y$&"X =:= Y."\\ \index{PROLOG =:=/2@\texttt{=:=/2}}
$x\not=y$&"X =\= Y."\\ \index{PROLOG =\=/2@\verb-=\=/2-}
$x\ge y$&"X >= Y"\\ \index{PROLOG >=/2@\texttt{>=/2}}
$x>y$&"X > Y" \index{PROLOG >/2@\texttt{>/2}}
\end{tabular}\end{center}

These operators have the obvious meaning:
\begin{LPNcodedisplay}
?- 2 < 4.
yes

?- 2 =< 4.
yes

?- 4 =< 4.
yes

?- 4=:=4.
yes

?- 4=\=5.
yes

?- 4=\=4.
no

?- 4 >= 4.
yes

?- 4 > 2.
yes
\end{LPNcodedisplay}


Moreover, they force both their right hand and left hand arguments to
be evaluated:
\begin{LPNcodedisplay}
?- 2 < 4+1.
yes

?- 2+1 < 4.
yes

?- 2+1 < 3+2.
yes
\end{LPNcodedisplay}


Note that "=:=" is different from "=", as the following examples show:
\begin{LPNcodedisplay}
?- 4=4.
yes

?- 2+2 =4.
no

?- 2+2 =:= 4.
yes
\end{LPNcodedisplay}
That is, "=" tries to unify its arguments; it does \LPNemph{not}
force arithmetic evaluation.  That's "=:="'s job.

Whenever we use these operators, we have to take care that any
variables are instantiated.  For example, all the following queries
lead to instantiation errors.
\begin{LPNcodedisplay}
?- X < 3.

?- 3 < Y.

?- X =:= X.
\end{LPNcodedisplay}
Moreover, variables have to be instantiated to \LPNemph{integers}.
The query
\begin{LPNcodedisplay}
?- X = 3, X < 4.
\end{LPNcodedisplay}
succeeds. But the query
\begin{LPNcodedisplay}
?- X = b, X < 4.
\end{LPNcodedisplay}
fails.

Ok, let's now look at an example which puts Prolog's abilities to
compare numbers to work.  We're going to define a predicate which
takes  a non-empty list of non-negative integers as its first
argument, and returns the maximum integer in the list as its last
argument.  Again, we'll use an accumulator.  As we work our way down
the list, the accumulator will keep track of the highest integer found
so far.  If we find a higher value, the accumulator will be updated to
this new value.  When we call the program, we set the accumulator to an
initial value of 0.

Here's the code.  Note that there are \LPNemph{two} recursive clauses:
\begin{LPNcodedisplay}
accMax([H|T],A,Max) :-
   H > A,
   accMax(T,H,Max).

accMax([H|T],A,Max) :-
   H =< A,
   accMax(T,A,Max).

accMax([],A,A).
\end{LPNcodedisplay}
The first clause tests if the head of the list is larger than the
largest value found so far.  If it is, we set the accumulator to this
new value, and then recursively work through the tail of the list.
The second clause applies when the head is less than or equal to the
accumulator; in this case we recursively work through the tail of the
list using the old accumulator value.  Finally, the base clause
unifies the second and third arguments; it gives the highest value we
found while going through the list to the last argument.

Here's an
example query:
\begin{LPNcodedisplay}
?- accMax([1,0,5,4],0,Max).
\end{LPNcodedisplay}
Here the first clause of "accMax" applies, resulting in the following
goal:
\begin{LPNcodedisplay}
?- accMax([0,5,4],1,Max).
\end{LPNcodedisplay}
Note the value of the accumulator has changed to 1.  Now the second
clause of "accMax" applies, as 0 (the next element of the list) is
smaller than 1, the value of the accumulator.
This process is repeated until
we reach the empty list:
\begin{LPNcodedisplay}
?- accMax([5,4],1,Max).

?- accMax([4],5,Max).

?- accMax([],5,Max).
\end{LPNcodedisplay}
Now the third clause applies, unifying the variable "Max"
with the value of the accumulator:
\begin{LPNcodedisplay}
Max = 5.
yes
\end{LPNcodedisplay}

Again, it's nice to define a predicate which calls this, and
initialises the accumulator.  But wait: what should we initialise the
accumulator to?  If you say 0, this means you are assuming that all
the numbers in the list are positive. But suppose we give
a list of negative integers as input. Then we would have
\begin{LPNcodedisplay}
?- accMax([-11,-2,-7,-4,-12],0,Max).

Max = 0
yes
\end{LPNcodedisplay}
This is \LPNemph{not} what we want: the biggest number on the list is
-2.  Our use of 0 as the initial value of the accumulator has ruined
everything, because it's bigger than any number on the list.

There's an easy way around this: since our input list will always be a
non-empty list of integers, simply initialise the accumulator to the
head of the list.  That way we guarantee that the accumulator is
initialised to a number on the list.  The following predicate does
this for us:
\begin{LPNcodedisplay}
max(List,Max) :-
     List = [H|_],
     accMax(List,H,Max).
\end{LPNcodedisplay}
So we can simply say:
\begin{LPNcodedisplay}
max([1,2,46,53,0],X).

X = 53
yes
\end{LPNcodedisplay}
And furthermore we have:
\begin{LPNcodedisplay}
max([-11,-2,-7,-4,-12],X).

X = -2
yes
\end{LPNcodedisplay}


\newpage
\section{Exercises}\label{SEC.L5.EXERCISES}

\begin{LPNexercise}{L5.EX1}How does Prolog respond to the following queries?
\begin{enumerate}
\item{}"X = 3*4."
\item{}"X is 3*4."
\item{}"4 is X."
\item{}"X = Y."
\item{}"3 is 1+2."
\item{}"3 is +(1,2)."
\item{}"3 is X+2."
\item{}"X is 1+2."
\item{}"1+2 is 1+2."
\item{}"is(X,+(1,2))."
\item{}"3+2 = +(3,2)."
\item{}"*(7,5) = 7*5."
\item{}"*(7,+(3,2)) = 7*(3+2)."
\item{}"*(7,(3+2)) = 7*(3+2)."
\item{}"7*3+2 = *(7,+(3,2))."
\item{}"*(7,(3+2)) = 7*(+(3,2))."
\end{enumerate}
\end{LPNexercise}
\begin{LPNexercise}{L5.EX2}\begin{enumerate}
\item{}Define a 2-place predicate "increment" that holds only when its
second argument is an integer one larger than its first argument.  For
example, "increment(4,5)" should hold, but "increment(4,6)"
should not.
\item{}Define a 3-place predicate "sum" that holds only when its third
argument is the sum of the first two arguments. For example,
"sum(4,5,9)" should hold, but "sum(4,6,12)" should
not.
\end{enumerate}
\end{LPNexercise}

\clearpage
\begin{LPNexercise}{L5.EX3}Write a predicate
"addone/2" whose first argument is a list of integers, and whose
second argument is the list of integers obtained by adding 1 to each
integer in the first list.  For example, the query
\begin{LPNcodedisplay}
?- addone([1,2,7,2],X).
\end{LPNcodedisplay}
should give
\begin{LPNcodedisplay}
X = [2,3,8,3].
\end{LPNcodedisplay}
\end{LPNexercise}

\begin{LPNexercise}{L5.EX4}Write a predicate
"rom2num/2" whose first argument is a Roman numeral represented as a list of atoms, and whose
second argument is a positive integer obtained by converting the roman numeral into a number.
For example, the query
\begin{LPNcodedisplay}
?- rom2num([i,v],X).
\end{LPNcodedisplay}
should give
\begin{LPNcodedisplay}
X = 4.
\end{LPNcodedisplay}
\end{LPNexercise}


\section{Practical Session}\label{SEC.L5.PRAXIS}



The purpose of Practical Session 5 is to help you get familiar with
Prolog's arithmetic capabilities, and to give you some further
practice in list manipulation. To this end, we suggest the following
programming exercises:

\begin{enumerate}
\item{}In the text we discussed the 3-place predicate "accMax"
which  returned the maximum of a list of integers.  By changing
the code slightly, turn this into a 3-place predicate "accMin"
which returns the \LPNemph{minimum} of a list of integers.
\item{}In mathematics, an n-dimensional vector is a list of numbers of
length n. For example, "[2,5,12]" is a 3-dimensional vector, and
"[45,27,3,-4,6]" is a 5-dimensional vector.  One of the basic
operations on vectors is \LPNemph{scalar multiplication}. In this
operation, every element of a vector is multiplied by some number.
For example, if we scalar multiply the 3-dimensional vector
"[2,7,4]" by "3" the result is the 3-dimensional vector
"[6,21,12]".


Write a 3-place predicate "scalarMult" whose first
argument is an integer, whose second argument is a list of integers,
and whose third argument is the result of scalar multiplying
the second argument by the first. For example, the query
\begin{LPNcodedisplay}
?- scalarMult(3,[2,7,4],Result).
\end{LPNcodedisplay}

should yield
\begin{LPNcodedisplay}
Result = [6,21,12]
\end{LPNcodedisplay}

\item{}Another fundamental operation on vectors is the \LPNemph{dot
product}.  This operation combines two vectors of the same dimension
and yields a number as a result. The operation is carried out as
follows: the corresponding elements of the two vectors are multiplied,
and the results added. For example, the dot product of "[2,5,6]"
and "[3,4,1]" is "6+20+6", that is, "32".  Write a
3-place predicate "dot" whose first argument is a list of
integers, whose second argument is a list of integers of the same
length as the first, and whose third argument is the dot product of
the first argument with the second.  For example, the query
\begin{LPNcodedisplay}
?- dot([2,5,6],[3,4,1],Result).
\end{LPNcodedisplay}

should yield
\begin{LPNcodedisplay}
Result = 32
\end{LPNcodedisplay}

\end{enumerate}
